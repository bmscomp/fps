\documentclass{beamer}
\usetheme{Madrid}
\usecolortheme{dolphin}
\usepackage{listings}
\usepackage{xcolor}
\usepackage{amsmath}

\definecolor{scalablue}{RGB}{0,96,172}
\definecolor{scalared}{RGB}{220,50,47}
\definecolor{scalagreen}{RGB}{133,153,0}

\lstdefinestyle{scala}{
  language=scala,
  basicstyle=\ttfamily\small,
  keywordstyle=\color{scalablue},
  commentstyle=\color{scalagreen},
  stringstyle=\color{scalared},
  showstringspaces=false,
  breaklines=true,
  tabsize=2,
  morekeywords={class, trait, object, extends, with, new, given, using, match, case, type, val, var, def, this, super, enum, export, transparent, inline, opaque, open}
}

\title{Functional Programming in Scala}
\subtitle{Classes, Objects, and Traits}
\author{Said BOUDJELDA}
\institute{Eferi}
\titlegraphic{\includegraphics[width=5cm]{logo}}
\date{\today}

\begin{document}

\frame{\titlepage}

\section{Introduction to Scala 3 OOP}
\begin{frame}[fragile]
\frametitle{Scala 3 OOP Overview}
\begin{itemize}
\item \textbf{Classes}: Templates for object creation
\item \textbf{Objects}: Singleton instances
\item \textbf{Traits}: Reusable behavior components

\begin{lstlisting}[style=scala]
// Class example
class Person(val name: String)

// Object example
object Logger:
  def log(msg: String) = println(msg)

// Trait example
trait Speaker:
  def speak(): String
\end{lstlisting}
\end{itemize}
\end{frame}

\begin{frame}[fragile]
\frametitle{Scala 3 vs Scala 2 Improvements}
\begin{itemize}
\item \textbf{Simpler syntax}:
\begin{lstlisting}[style=scala]
// Old
class Foo { ... }
// New
class Foo: ...
\end{lstlisting}

\item \textbf{New modifiers}:
\begin{lstlisting}[style=scala]
open class Parent  // Explicitly extensible
transparent trait Logger  // More precise type checking
\end{lstlisting}

\item \textbf{Trait parameters}:
\begin{lstlisting}[style=scala]
trait Config(env: String)  // New in Scala 3
\end{lstlisting}
\end{itemize}
\end{frame}

\section{Classes in Depth}
\begin{frame}[fragile]
\frametitle{Class Constructors}
\begin{block}{Primary Constructor}
\begin{lstlisting}[style=scala]
class Person(
  val name: String,      // Public read-only field
  var age: Int,         // Public mutable field
  private val id: Int   // Private field
):
  def this(name: String) = this(name, 0, 0)  // Auxiliary
\end{lstlisting}
\end{block}

\begin{itemize}
\item Parameters in class declaration become constructor parameters
\item \texttt{val} creates immutable field, \texttt{var} mutable
\item Auxiliary constructors must call primary constructor
\end{itemize}
\end{frame}

\begin{frame}[fragile]
\frametitle{Class Inheritance}
\begin{lstlisting}[style=scala]
open class Animal(val name: String):
  def makeSound(): String = "Some sound"

class Dog(name: String) extends Animal(name):
  override def makeSound() = "Woof!"
  def fetch() = "Fetching..."
\end{lstlisting}

Key points:
\begin{itemize}
\item \texttt{open} required for extension (new in Scala 3)
\item \texttt{override} modifier required
\item Single inheritance only (use traits for multiple)
\end{itemize}
\end{frame}

\begin{frame}[fragile]
\frametitle{Case Classes}
\begin{lstlisting}[style=scala]
case class Person(
  name: String,
  age: Int,
  address: Address = Address.default
)

// Automatically gets:
// 1. equals/hashCode
// 2. toString
// 3. copy method
// 4. companion object with apply
// 5. pattern matching support
\end{lstlisting}

Usage examples:
\begin{lstlisting}[style=scala]
val p1 = Person("Alice", 30)
val p2 = p1.copy(age = 31)  // Non-destructive update
\end{lstlisting}
\end{frame}

\begin{frame}[fragile]
\frametitle{Abstract Classes}
\begin{lstlisting}[style=scala]
abstract class Shape(val color: String):
  def area: Double      // Abstract method
  def perimeter: Double // Abstract method
  def describe = s"$color shape"  // Concrete method

class Circle(color: String, radius: Double) 
  extends Shape(color):
    def area = math.Pi * radius * radius
    def perimeter = 2 * math.Pi * radius
\end{lstlisting}

Characteristics:
\begin{itemize}
\item Cannot be instantiated
\item Can contain both abstract and concrete members
\item Single inheritance still applies
\end{itemize}
\end{frame}

\section{Objects in Depth}
\begin{frame}[fragile]
\frametitle{Singleton Objects}
\begin{lstlisting}[style=scala]
object MathConstants:
  val PI = 3.1415926535
  val E = 2.7182818284
  def square(x: Double) = x * x

// Usage:
MathConstants.PI
MathConstants.square(5)
\end{lstlisting}

Key uses:
\begin{itemize}
\item Constants and utility methods
\item Factory methods
\item Entry points for applications
\end{itemize}
\end{frame}

\begin{frame}[fragile]
\frametitle{Companion Objects}
\begin{lstlisting}[style=scala]
class BankAccount private (val balance: Double):
  // Instance members here

object BankAccount:  // Companion
  def apply(initial: Double) = 
    new BankAccount(initial)
    
  def fromString(s: String): Option[BankAccount] = 
    s.toDoubleOption.map(new BankAccount(_))
\end{lstlisting}

Benefits:
\begin{itemize}
\item Access to private class members
\item Logical grouping of factory methods
\item Alternative constructors
\end{itemize}
\end{frame}

\begin{frame}[fragile]
\frametitle{Case Class Companions}
\begin{lstlisting}[style=scala]
case class Email(user: String, domain: String)

// Generated companion includes:
object Email:
  // Factory method
  def apply(user: String, domain: String) = 
    new Email(user, domain)
    
  // Extractor for pattern matching
  def unapply(email: Email): Option[(String, String)] = 
    Some((email.user, email.domain))
\end{lstlisting}

Usage:
\begin{lstlisting}[style=scala]
Email("user", "example.com")  // No 'new' needed
email match
  case Email(u, d) => println(s"User: $u, Domain: $d")
\end{lstlisting}
\end{frame}

\begin{frame}[fragile]
\frametitle{Object Inheritance}
\begin{lstlisting}[style=scala]
trait JsonSerializer:
  def toJson: String

object DefaultSerializer extends JsonSerializer:
  def toJson = "{}"

// Objects can extend classes/traits
// But cannot be extended themselves
\end{lstlisting}

Limitations:
\begin{itemize}
\item Objects are final (cannot be extended)
\item Can mix in multiple traits
\item Useful for implementing type classes
\end{itemize}
\end{frame}

\section{Traits in Depth}
\begin{frame}[fragile]
\frametitle{Trait Basics}
\begin{lstlisting}[style=scala]
trait Logger:
  def log(msg: String): Unit  // Abstract method
  def info(msg: String) = log(s"INFO: $msg")  // Concrete
  def warn(msg: String) = log(s"WARN: $msg")  // Concrete

class ConsoleLogger extends Logger:
  def log(msg: String) = println(msg)  // Implement abstract
\end{lstlisting}

Characteristics:
\begin{itemize}
\item Can contain abstract and concrete members
\item Multiple inheritance allowed
\item Cannot have constructor parameters (pre-Scala 3)
\end{itemize}
\end{frame}

\begin{frame}[fragile]
\frametitle{Trait Parameters (Scala 3)}
\begin{lstlisting}[style=scala]
trait Greeting(val prefix: String):
  def greet(name: String) = s"$prefix $name"

class FormalGreeter extends Greeting("Dear")
class CasualGreeter extends Greeting("Hey")

// Usage:
FormalGreeter().greet("Alice")  // "Dear Alice"
CasualGreeter().greet("Bob")    // "Hey Bob"
\end{lstlisting}

Advantages:
\begin{itemize}
\item Parameterized behavior without abstract members
\item Evaluated exactly once when mixed in
\item Alternative to constructor parameters in traits
\end{itemize}
\end{frame}

\begin{frame}[fragile]
\frametitle{Stackable Traits Pattern}
\begin{lstlisting}[style=scala]
abstract class IntQueue:
  def get(): Int
  def put(x: Int): Unit

trait Doubling extends IntQueue:
  abstract override def put(x: Int) = 
    super.put(2 * x)

trait Incrementing extends IntQueue:
  abstract override def put(x: Int) = 
    super.put(x + 1)

class BasicQueue extends IntQueue:
  private val buf = collection.mutable.ArrayBuffer.empty[Int]
  def get() = buf.remove(0)
  def put(x: Int) = buf += x
\end{lstlisting}
\end{frame}

\begin{frame}[fragile]
\frametitle{Stackable Traits Usage}
\begin{lstlisting}[style=scala]
val q1 = new BasicQueue with Doubling
q1.put(10)  // Adds 20
q1.get()    // Returns 20

val q2 = new BasicQueue with Incrementing
q2.put(10)  // Adds 11
q2.get()    // Returns 11

val q3 = new BasicQueue with Doubling with Incrementing
q3.put(10)  // Adds 22 (10*2 then +1)
q3.get()    // Returns 22

val q4 = new BasicQueue with Incrementing with Doubling
q4.put(10)  // Adds 21 (10+1 then *2)
q4.get()    // Returns 21
\end{lstlisting}
\end{frame}

\begin{frame}[fragile]
\frametitle{Self Types}
\begin{lstlisting}[style=scala]
trait UserRepository:
  def findUser(id: Int): User

trait UserService { self: UserRepository =>  // Self-type
  def getUser(id: Int): User = 
    findUser(id).orElse(defaultUser)
}

// Must mix in UserRepository
class UserServiceImpl extends UserService with UserRepository:
  def findUser(id: Int) = ...
\end{lstlisting}

Purpose:
\begin{itemize}
\item Declare trait dependencies without inheritance
\item Enforce required mixins
\item Avoid circular dependencies
\end{itemize}
\end{frame}

\section{Advanced OOP Patterns}
\begin{frame}[fragile]
\frametitle{Type Class Pattern}
\begin{lstlisting}[style=scala]
// 1. Type class definition
trait JsonWriter[A]:
  def write(value: A): Json

// 2. Type class instances
object JsonWriterInstances:
  given JsonWriter[String] with
    def write(s: String) = JsonString(s)
    
  given JsonWriter[Int] with
    def write(n: Int) = JsonNumber(n)

// 3. Interface
object Json:
  def toJson[A](value: A)(using writer: JsonWriter[A]) = 
    writer.write(value)
\end{lstlisting}
\end{frame}

\begin{frame}[fragile]
\frametitle{Extension Methods}
\begin{lstlisting}[style=scala]
trait StringExtensions:
  extension (s: String)
    def toTitleCase: String = 
      s.split(" ").map(_.capitalize).mkString(" ")
      
    def encrypt(shift: Int): String = 
      s.map(c => (c + shift).toChar)

// Usage:
import StringExtensions.*
"hello world".toTitleCase  // "Hello World"
"abc".encrypt(1)          // "bcd"
\end{lstlisting}

Benefits:
\begin{itemize}
\item Add methods to existing types
\item More discoverable than implicit classes
\item Group related extensions together
\end{itemize}
\end{frame}

\begin{frame}[fragile]
\frametitle{Factory Pattern with Companion}
\begin{lstlisting}[style=scala]
sealed abstract class DatabaseConfig

object DatabaseConfig:
  // Private implementations
  private case class PostgresConfig(url: String) 
    extends DatabaseConfig
    
  private case class MongoConfig(uri: String) 
    extends DatabaseConfig

  // Factory methods
  def postgres(url: String): DatabaseConfig = 
    PostgresConfig(url)
    
  def mongo(uri: String): DatabaseConfig = 
    MongoConfig(uri)
\end{lstlisting}
\end{frame}

\begin{frame}[fragile]
\frametitle{Value Classes}
\begin{lstlisting}[style=scala]
class Meter(val value: Double) extends AnyVal:
  def +(m: Meter): Meter = new Meter(value + m.value)
  def toKm: Kilometer = new Kilometer(value / 1000)

class Kilometer(val value: Double) extends AnyVal:
  def toMeters: Meter = new Meter(value * 1000)

// Usage:
val distance = Meter(500) + Meter(300)
val inKm = distance.toKm  // No runtime overhead
\end{lstlisting}

Characteristics:
\begin{itemize}
\item Extends \texttt{AnyVal}
\item No runtime allocation overhead
\item Type safety without performance cost
\end{itemize}
\end{frame}

\section{Best Practices}
\begin{frame}[fragile]
\frametitle{Class Design Principles}
\begin{itemize}
\item \textbf{Favor immutability}:
\begin{lstlisting}[style=scala]
// Prefer
case class Point(x: Double, y: Double)

// Over
class MutablePoint(var x: Double, var y: Double)
\end{lstlisting}

\item \textbf{Small, focused classes}:
\begin{itemize}
\item Single Responsibility Principle
\item 50-100 lines max
\end{itemize}

\item \textbf{Document invariants}:
\begin{lstlisting}[style=scala]
class NonEmptyList[A](val head: A, val tail: List[A]):
  require(tail != null, "Tail cannot be null")
\end{lstlisting}
\end{itemize}
\end{frame}

\begin{frame}[fragile]
\frametitle{Trait Design Guidelines}
\begin{itemize}
\item \textbf{Keep traits focused}:
\begin{lstlisting}[style=scala]
// Good
trait Logging
trait Authentication
trait DatabaseAccess

// Bad
trait ControllerUtilities  // Too vague
\end{lstlisting}

\item \textbf{Use sealed traits for closed hierarchies}:
\begin{lstlisting}[style=scala]
sealed trait Response
case class Success(data: String) extends Response
case class Failure(error: String) extends Response
\end{lstlisting}

\item \textbf{Linearization matters}:
\begin{itemize}
\item Traits are stacked last-to-first
\item Put fundamental traits first
\end{itemize}
\end{itemize}
\end{frame}

\begin{frame}[fragile]
\frametitle{Object Usage Patterns}
\begin{itemize}
\item \textbf{For type class instances}:
\begin{lstlisting}[style=scala]
object JsonWriters:
  given JsonWriter[Int] = ...
  given JsonWriter[String] = ...
\end{lstlisting}

\item \textbf{As modules}:
\begin{lstlisting}[style=scala]
object DatabaseModule:
  def connect(config: Config) = ...
  def query(sql: String) = ...
\end{lstlisting}

\item \textbf{For entry points}:
\begin{lstlisting}[style=scala]
@main def runApp(): Unit =
  println("Application started")
\end{lstlisting}
\end{itemize}
\end{frame}

\section{Conclusion}
\begin{frame}
\frametitle{Key Takeaways}
\begin{itemize}
\item \textbf{Classes} are templates for objects with state and behavior
\item \textbf{Objects} provide singleton instances and utilities
\item \textbf{Traits} enable flexible composition of behavior
\item Scala 3 adds:
\begin{itemize}
\item Trait parameters
\item Open classes
\item Improved syntax
\end{itemize}
\item Combine these features for clean, modular designs
\end{itemize}
\end{frame}

\begin{frame}
\frametitle{Further Learning}
\begin{itemize}
\item Official Scala 3 Documentation
\item "Programming in Scala, 5th Edition" (Odersky et al.)
\item "Scala with Cats" (Noel Welsh and Dave Gurnell)
\item Scala Exercises (https://www.scala-exercises.org/)
\item Contribute to open-source Scala projects
\end{itemize}
\end{frame}

\end{document}
