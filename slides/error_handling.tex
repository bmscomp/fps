\documentclass{beamer}
\usepackage[utf8]{inputenc}
\usepackage{booktabs}
\usepackage{qrcode}
\usepackage{graphicx}
\usepackage{listings}
\usepackage{xcolor}
\usepackage{multirow}
\usepackage{array}

% Define Scala syntax highlighting
\lstdefinestyle{scalaStyle}{
  language=Scala,
  basicstyle=\ttfamily\small,
  keywordstyle=\color{blue},
  commentstyle=\color{brown},
  stringstyle=\color{red},
  showstringspaces=false,
  breaklines=true,
  frame=single,
  numbers=left,
  numberstyle=\tiny\color{gray}
}

\definecolor{scalared}{RGB}{215,50,58}
\definecolor{darkblue}{RGB}{0,0,139}
\definecolor{pureblue}{RGB}{0,82,155}
\definecolor{impurered}{RGB}{186,0,13}
\definecolor{titleblue}{RGB}{0,82,155}
\definecolor{alertred}{RGB}{186,0,13}
\definecolor{darkgray}{RGB}{80,80,80}

\usetheme{Madrid}
\usecolortheme{default}

\title[Scala Error Handling]
{Error Handling in Scala 3}

\subtitle{From Exceptions to Functional Error Management}

\author[Said BOUDJELDA]
{Said BOUDJELDA}

\institute[efrei]
{
  Senior Software Engineer @SCIAM\\
  Email : mohamed-said.boudjelda@intervenants.efrei.net \\ 
  Follow me on GitHub @bmscomp
}

\date[efrei 2025]
{Course, May 2025}

\logo{\includegraphics[height=0.6cm]{logo}}

\begin{document}

\frame{\titlepage}

%---------------------------------------------------------
\begin{frame}
\frametitle{Overview}

Error handling is crucial in any programming language. Scala 3 offers multiple approaches:

\begin{itemize}
\item \textbf{Traditional}: Exceptions and try-catch blocks
\item \textbf{Functional}: Option, Either, Try types
\item \textbf{Modern}: Union types and improved pattern matching
\end{itemize}

We'll explore evolution from imperative to functional error handling.

\end{frame}

%---------------------------------------------------------
\begin{frame}[fragile]
\frametitle{Traditional Exception Handling}
\framesubtitle{The Old Way}

\begin{lstlisting}[style=scalaStyle]
// Traditional Java-style exception handling
def divide(a: Int, b: Int): Int = {
  if (b == 0) 
    throw new ArithmeticException("Division by zero")
  else 
    a / b
}

try {
  val result = divide(10, 0)
  println(s"Result: $result")
} catch {
  case e: ArithmeticException => 
    println(s"Error: ${e.getMessage}")
} finally {
  println("Cleanup operations")
}
\end{lstlisting}

\end{frame}

%---------------------------------------------------------
\begin{frame}[fragile]
\frametitle{Problems with Exceptions}

\begin{itemize}
\item \textbf{Not type-safe}: Exceptions are not tracked in method signatures
\item \textbf{Control flow}: Breaks normal program flow
\item \textbf{Performance}: Stack unwinding is expensive
\item \textbf{Composition}: Hard to compose operations that might fail
\end{itemize}

\begin{lstlisting}[style=scalaStyle]
// Signature doesn't tell us this method can fail
def parseNumber(s: String): Int = s.toInt // Can throw!

// Callers might forget to handle exceptions
val num = parseNumber("not-a-number") // Runtime crash!
\end{lstlisting}

\end{frame}

%---------------------------------------------------------
\begin{frame}[fragile]
\frametitle{Option Type - Handling Null Values}
\framesubtitle{Functional Approach}

\begin{lstlisting}[style=scalaStyle]
// Option represents optional values - Some or None
def safeDivide(a: Int, b: Int): Option[Int] = 
  if (b == 0) None else Some(a / b)

// Pattern matching
safeDivide(10, 2) match {
  case Some(result) => println(s"Result: $result")
  case None => println("Division by zero")
}

// Functional operations
val result = safeDivide(10, 2)
  .map(_ * 2)           // Transform if present
  .filter(_ > 5)        // Filter condition
  .getOrElse(0)         // Default value

println(result) // 10
\end{lstlisting}

\end{frame}

%---------------------------------------------------------
\begin{frame}[fragile]
\frametitle{Option - Advanced Operations}

\begin{lstlisting}[style=scalaStyle]
// Chaining operations that might fail
def parseAge(s: String): Option[Int] = 
  try Some(s.toInt) catch case _ => None

def validateAge(age: Int): Option[Int] = 
  if (age >= 0 && age <= 150) Some(age) else None

// Composition using flatMap
def processAge(input: String): Option[String] = 
  parseAge(input)
    .flatMap(validateAge)
    .map(age => s"Valid age: $age")

println(processAge("25"))    // Some(Valid age: 25)
println(processAge("200"))   // None
println(processAge("abc"))   // None
\end{lstlisting}

\end{frame}

%---------------------------------------------------------
\begin{frame}[fragile]
\frametitle{Either Type - Rich Error Information}
\framesubtitle{Left = Error, Right = Success}

\begin{lstlisting}[style=scalaStyle]
sealed trait AppError
case class ValidationError(msg: String) extends AppError
case class ParseError(msg: String) extends AppError

def parseAndValidateAge(s: String): Either[AppError, Int] = 
  try {
    val age = s.toInt
    if (age >= 0 && age <= 150) 
      Right(age)
    else 
      Left(ValidationError(s"Invalid age: $age"))
  } catch {
    case _: NumberFormatException => 
      Left(ParseError(s"Not a number: $s"))
  }

parseAndValidateAge("25") match {
  case Right(age) => println(s"Valid age: $age")
  case Left(error) => println(s"Error: $error")
}
\end{lstlisting}

\end{frame}

%---------------------------------------------------------
\begin{frame}[fragile]
\frametitle{Either - Functional Operations}

\begin{lstlisting}[style=scalaStyle]
// Either is right-biased in Scala 2.12+
val result = parseAndValidateAge("25")
  .map(_ * 2)                    // Only if Right
  .flatMap(age => 
    if (age < 100) Right(s"Young: $age")
    else Left(ValidationError("Too old")))

// For-comprehension with Either
def processUser(name: String, ageStr: String) = 
  for {
    age <- parseAndValidateAge(ageStr)
    validName <- if (name.nonEmpty) Right(name) 
                else Left(ValidationError("Empty name"))
  } yield User(validName, age)

case class User(name: String, age: Int)

println(processUser("John", "25")) // Right(User(John,25))
\end{lstlisting}

\end{frame}

%---------------------------------------------------------
\begin{frame}[fragile]
\frametitle{Try Type - Exception Wrapping}

\begin{lstlisting}[style=scalaStyle]
import scala.util.{Try, Success, Failure}

// Try wraps operations that might throw exceptions
def safeParse(s: String): Try[Int] = Try(s.toInt)

def safeFileRead(filename: String): Try[String] = 
  Try(scala.io.Source.fromFile(filename).mkString)

// Pattern matching
safeParse("123") match {
  case Success(num) => println(s"Parsed: $num")
  case Failure(ex) => println(s"Failed: ${ex.getMessage}")
}

// Functional operations
val result = safeParse("42")
  .map(_ * 2)
  .recover { case _: NumberFormatException => 0 }
  .get

println(result) // 84
\end{lstlisting}

\end{frame}

%---------------------------------------------------------
\begin{frame}[fragile]
\frametitle{Scala 3 Union Types for Errors}
\framesubtitle{Modern Approach}

\begin{lstlisting}[style=scalaStyle]
// Union types in Scala 3
type ParseResult = Int | String

def parseNumber(s: String): ParseResult = 
  try s.toInt
  catch case _: NumberFormatException => s"Invalid: $s"

// Pattern matching with union types
parseNumber("42") match {
  case num: Int => println(s"Parsed: $num")
  case error: String => println(s"Error: $error")
}

// More complex union types
type Result[T] = T | Exception

def divide(a: Int, b: Int): Result[Double] = 
  if (b == 0) ArithmeticException("Division by zero")
  else a.toDouble / b
\end{lstlisting}

\end{frame}

%---------------------------------------------------------
\begin{frame}[fragile]
\frametitle{Error Accumulation with Validated}
\framesubtitle{Collecting Multiple Errors}

\begin{lstlisting}[style=scalaStyle]
// Using cats library for error accumulation
import cats.data.Validated
import cats.syntax.all._

type ValidationResult[T] = Validated[List[String], T]

def validateName(name: String): ValidationResult[String] = 
  if (name.nonEmpty) name.valid
  else List("Name cannot be empty").invalid

def validateAge(age: Int): ValidationResult[Int] = 
  if (age >= 0 && age <= 150) age.valid
  else List(s"Invalid age: $age").invalid

// Accumulate all errors
(validateName(""), validateAge(200)).mapN(User.apply) match {
  case Validated.Valid(user) => println(s"User: $user")
  case Validated.Invalid(errors) => 
    println(s"Errors: ${errors.mkString(", ")}")
}
// Output: Errors: Name cannot be empty, Invalid age: 200
\end{lstlisting}

\end{frame}

%---------------------------------------------------------
\begin{frame}[fragile]
\frametitle{Custom Error ADTs}
\framesubtitle{Algebraic Data Types for Errors}

\begin{lstlisting}[style=scalaStyle]
// Define comprehensive error hierarchy
sealed trait DatabaseError extends Exception
case class ConnectionError(msg: String) extends DatabaseError
case class QueryError(sql: String, msg: String) extends DatabaseError
case class TimeoutError(seconds: Int) extends DatabaseError

sealed trait ValidationError extends Exception
case class InvalidEmail(email: String) extends ValidationError
case class InvalidPassword(reason: String) extends ValidationError

// Combine different error types
type AppError = DatabaseError | ValidationError

def createUser(email: String, password: String): Either[AppError, User] = 
  for {
    validEmail <- validateEmail(email)
    validPassword <- validatePassword(password)
    user <- saveToDatabase(validEmail, validPassword)
  } yield user
\end{lstlisting}

\end{frame}

%---------------------------------------------------------
\begin{frame}[fragile]
\frametitle{Error Handling with For-Comprehensions}

\begin{lstlisting}[style=scalaStyle]
// Sequential error handling
def processOrder(): Either[String, Order] = 
  for {
    user <- findUser("john@example.com")
    product <- findProduct("laptop")
    inventory <- checkInventory(product.id)
    order <- createOrder(user, product) if inventory > 0
  } yield order

// With custom error types
def processOrderAdvanced(): Either[AppError, Order] = 
  for {
    user <- findUser("john@example.com")
              .toRight(UserNotFound("john@example.com"))
    product <- findProduct("laptop")
                .toRight(ProductNotFound("laptop"))
    _ <- validateInventory(product)
    order <- saveOrder(user, product)
  } yield order
\end{lstlisting}

\end{frame}

%---------------------------------------------------------
\begin{frame}[fragile]
\frametitle{Resource Management with Using}
\framesubtitle{Scala 3 Automatic Resource Management}

\begin{lstlisting}[style=scalaStyle]
import scala.util.Using

// Automatic resource cleanup
def readFileContent(filename: String): Try[String] = 
  Using(scala.io.Source.fromFile(filename)) { source =>
    source.getLines().mkString("\n")
  }

// Multiple resources
def copyFile(from: String, to: String): Try[Unit] = 
  Using.Manager { use =>
    val source = use(scala.io.Source.fromFile(from))
    val writer = use(java.io.PrintWriter(to))
    source.getLines().foreach(writer.println)
  }

// Resource is automatically closed even if exception occurs
readFileContent("config.txt") match {
  case Success(content) => println(content)
  case Failure(ex) => println(s"Failed to read: ${ex.getMessage}")
}
\end{lstlisting}

\end{frame}

%---------------------------------------------------------
\begin{frame}[fragile]
\frametitle{Error Recovery Strategies}

\begin{lstlisting}[style=scalaStyle]
// Retry mechanism
def withRetry[T](maxAttempts: Int)(operation: () => Try[T]): Try[T] = 
  operation() match {
    case success @ Success(_) => success
    case Failure(_) if maxAttempts > 1 => 
      withRetry(maxAttempts - 1)(operation)
    case failure => failure
  }

// Circuit breaker pattern
class CircuitBreaker(failureThreshold: Int) {
  private var failureCount = 0
  private var state: State = Closed
  
  def execute[T](operation: () => T): Try[T] = 
    state match {
      case Closed => Try(operation()).recoverWith(handleFailure)
      case Open => Failure(new Exception("Circuit breaker open"))
    }
}
\end{lstlisting}

\end{frame}

%---------------------------------------------------------
\begin{frame}[fragile]
\frametitle{Async Error Handling with Future}

\begin{lstlisting}[style=scalaStyle]
import scala.concurrent.Future
import scala.concurrent.ExecutionContext.Implicits.global

// Async operations with error handling
def fetchUser(id: Int): Future[Either[String, User]] = 
  Future {
    // Simulate network call
    Thread.sleep(100)
    if (id > 0) Right(User(s"user$id", 25))
    else Left("Invalid user ID")
  }

// Combine async operations
def getUserProfile(id: Int): Future[Either[String, Profile]] = 
  for {
    userResult <- fetchUser(id)
    profile <- userResult match {
      case Right(user) => fetchProfile(user.id)
      case Left(error) => Future.successful(Left(error))
    }
  } yield profile

// Error recovery
fetchUser(-1).recover {
  case ex: Exception => Left(s"Network error: ${ex.getMessage}")
}
\end{lstlisting}

\end{frame}

%---------------------------------------------------------
\begin{frame}[fragile]
\frametitle{Monadic Error Handling}
\framesubtitle{Composing Operations}

\begin{lstlisting}[style=scalaStyle]
// Error monad for chaining operations
case class Result[+T](value: Either[String, T]) {
  def map[U](f: T => U): Result[U] = 
    Result(value.map(f))
  
  def flatMap[U](f: T => Result[U]): Result[U] = 
    value match {
      case Right(v) => f(v)
      case Left(e) => Result(Left(e))
    }
}

object Result {
  def success[T](value: T): Result[T] = Result(Right(value))
  def failure[T](error: String): Result[T] = Result(Left(error))
}

// Usage
val computation = for {
  a <- Result.success(10)
  b <- Result.success(20)
  c <- if (b != 0) Result.success(a / b) else Result.failure("Division by zero")
} yield c
\end{lstlisting}

\end{frame}

%---------------------------------------------------------
\begin{frame}[fragile]
\frametitle{Error Handling Best Practices}

\begin{itemize}
\item \textbf{Use types}: Make errors explicit in function signatures
\item \textbf{Avoid exceptions}: For predictable failures, use Option/Either
\item \textbf{Fail fast}: Validate inputs early
\item \textbf{Error accumulation}: Collect all validation errors
\item \textbf{Recovery}: Provide fallback mechanisms
\end{itemize}

\begin{lstlisting}[style=scalaStyle]
// Good: Error is explicit in return type
def parseConfig(file: String): Either[ConfigError, Config]

// Bad: Exception not visible in signature
def parseConfig(file: String): Config // throws ConfigException

// Good: Accumulate validation errors
def validateUser(data: UserData): ValidatedNel[Error, User]

// Bad: Stop at first error
def validateUser(data: UserData): Either[Error, User]
\end{lstlisting}

\end{frame}

%---------------------------------------------------------
\begin{frame}[fragile]
\frametitle{Performance Considerations}

\begin{lstlisting}[style=scalaStyle]
// Option/Either allocation overhead
def heavyComputation(): Option[Int] = {
  // Avoid creating Option for every intermediate step
  val intermediate = computeValue()
  if (isValid(intermediate)) Some(intermediate) else None
}

// Use specialized collections for performance
import scala.collection.mutable

// For high-performance scenarios, consider using:
// - Specialized Option types (OptionalInt, etc.)
// - Custom Result types with value classes
// - Unboxed union types in Scala 3

value class UserId(val value: Int) extends AnyVal
type UserResult = UserId | String // Union type - no boxing!

def findUser(id: Int): UserResult = 
  if (id > 0) UserId(id) else "Invalid ID"
\end{lstlisting}

\end{frame}

%---------------------------------------------------------
\begin{frame}[fragile]
\frametitle{Scala 3 Improvements}
\framesubtitle{New Features for Error Handling}

\begin{lstlisting}[style=scalaStyle]
// Union types for error handling
type ParseError = NumberFormatException | IllegalArgumentException

// Improved pattern matching
def handleError(error: Throwable): String = error match {
  case _: NumberFormatException => "Invalid number format"
  case _: IllegalArgumentException => "Invalid argument"
  case _ => "Unknown error"
}

// Enums for error codes
enum ErrorCode {
  case ValidationFailed, NetworkTimeout, DatabaseError
  
  def message: String = this match {
    case ValidationFailed => "Input validation failed"
    case NetworkTimeout => "Network request timed out"
    case DatabaseError => "Database operation failed"
  }
}
\end{lstlisting}

\end{frame}

%---------------------------------------------------------
\begin{frame}[fragile]
\frametitle{Error Boundary Pattern}

\begin{lstlisting}[style=scalaStyle]
// Error boundary for isolating failures
trait ErrorBoundary[F[_]] {
  def handle[A](fa: F[A])(recover: Throwable => A): F[A]
}

// Implementation for Future
given ErrorBoundary[Future] with {
  def handle[A](fa: Future[A])(recover: Throwable => A): Future[A] = 
    fa.recover { case ex => recover(ex) }
}

// Usage
def safeOperation[F[_]: ErrorBoundary](computation: F[String]): F[String] = 
  summon[ErrorBoundary[F]].handle(computation) { ex =>
    s"Operation failed: ${ex.getMessage}"
  }

// Boundary isolates errors from propagating up
val result = safeOperation(Future(throw new Exception("Boom!")))
// Future will contain "Operation failed: Boom!" instead of exception
\end{lstlisting}

\end{frame}

%---------------------------------------------------------
\begin{frame}[fragile]
\frametitle{Testing Error Scenarios}

\begin{lstlisting}[style=scalaStyle]
import org.scalatest.flatspec.AnyFlatSpec
import org.scalatest.matchers.should.Matchers

class ErrorHandlingSpec extends AnyFlatSpec with Matchers {
  
  "safeDivide" should "return None for division by zero" in {
    safeDivide(10, 0) shouldBe None
  }
  
  it should "return Some for valid division" in {
    safeDivide(10, 2) shouldBe Some(5)
  }
  
  "parseAndValidateAge" should "accumulate multiple errors" in {
    val result = validateUser(UserData("", -5))
    result.isInvalid shouldBe true
    result.swap.getOrElse(Nil) should contain allOf(
      "Name cannot be empty",
      "Invalid age: -5"
    )
  }
}
\end{lstlisting}

\end{frame}

%---------------------------------------------------------
\begin{frame}[fragile]
\frametitle{Migration Strategy}
\framesubtitle{From Exceptions to Functional}

\begin{lstlisting}[style=scalaStyle]
// Phase 1: Wrap existing exception-throwing code
def legacyOperation(): String = throw new RuntimeException("Legacy!")

def wrappedLegacy(): Try[String] = Try(legacyOperation())

// Phase 2: Introduce Either for domain errors
def improvedOperation(): Either[String, String] = 
  wrappedLegacy().toEither.left.map(_.getMessage)

// Phase 3: Use custom error types
sealed trait DomainError
case class LegacyError(msg: String) extends DomainError

def modernOperation(): Either[DomainError, String] = 
  improvedOperation().left.map(LegacyError.apply)

// Phase 4: Full functional error handling
def fullyFunctional(): Either[DomainError, String] = 
  Right("Success!") // No exceptions thrown
\end{lstlisting}

\end{frame}

%---------------------------------------------------------
\begin{frame}[fragile]
\frametitle{Real-World Example: HTTP Client}

\begin{lstlisting}[style=scalaStyle]
import sttp.client3._

sealed trait HttpError
case class NetworkError(cause: Throwable) extends HttpError
case class InvalidResponse(code: Int, body: String) extends HttpError
case class ParseError(json: String, cause: Throwable) extends HttpError

def fetchUser(id: Int): IO[Either[HttpError, User]] = {
  val request = basicRequest
    .get(uri"https://api.example.com/users/$id")
    .response(asString)
  
  for {
    response <- request.send(backend).attempt.map(_.left.map(NetworkError.apply))
    user <- response match {
      case Right(resp) if resp.code.isSuccess => 
        parseJson[User](resp.body).leftMap(ParseError(resp.body, _))
      case Right(resp) => 
        Left(InvalidResponse(resp.code.code, resp.body))
      case Left(error) => Left(error)
    }
  } yield user
}
\end{lstlisting}

\end{frame}

%---------------------------------------------------------
\begin{frame}[fragile]
\frametitle{Error Handling in Web Applications}

\begin{lstlisting}[style=scalaStyle]
// Using Tapir for HTTP API error handling
import sttp.tapir._

sealed trait ApiError
case class ValidationError(field: String, message: String) extends ApiError
case class NotFoundError(resource: String, id: String) extends ApiError
case class ServerError(message: String) extends ApiError

val getUserEndpoint = endpoint.get
  .in("users" / path[String]("id"))
  .out(jsonBody[User])
  .errorOut(oneOf[ApiError](
    oneOfVariant(statusCode(StatusCode.BadRequest).and(jsonBody[ValidationError])),
    oneOfVariant(statusCode(StatusCode.NotFound).and(jsonBody[NotFoundError])),
    oneOfVariant(statusCode(StatusCode.InternalServerError).and(jsonBody[ServerError]))
  ))

// Implementation automatically maps errors to HTTP responses
def getUser(id: String): IO[Either[ApiError, User]] = 
  userService.findById(id).map {
    case Some(user) => Right(user)
    case None => Left(NotFoundError("user", id))
  }
\end{lstlisting}

\end{frame}

%---------------------------------------------------------
\begin{frame}[fragile]
\frametitle{Monitoring and Observability}

\begin{lstlisting}[style=scalaStyle]
// Error tracking with structured logging
import org.slf4j.LoggerFactory
import io.circe.syntax._

val logger = LoggerFactory.getLogger(this.getClass)

def processWithLogging[A](operation: String)(thunk: => Either[AppError, A]): Either[AppError, A] = {
  val startTime = System.currentTimeMillis()
  
  thunk match {
    case Right(result) => 
      logger.info(s"$operation completed successfully in ${System.currentTimeMillis() - startTime}ms")
      Right(result)
    
    case Left(error) => 
      logger.error(s"$operation failed after ${System.currentTimeMillis() - startTime}ms", 
        Map("operation" -> operation, "error" -> error.toString).asJson.noSpaces)
      Left(error)
  }
}

// Usage
val result = processWithLogging("user-creation") {
  createUser("john@example.com", "password123")
}
\end{lstlisting}

\end{frame}

%---------------------------------------------------------
\begin{frame}[fragile]
\frametitle{Error Handling Patterns Summary}

\begin{table}[ht]
\centering
\small
\begin{tabular}{|l|l|l|l|}
\hline
\textbf{Pattern} & \textbf{Use Case} & \textbf{Pros} & \textbf{Cons} \\
\hline
Try-Catch & Legacy code & Familiar & Not type-safe \\
\hline
Option & Null safety & Simple & No error info \\
\hline
Either & Rich errors & Type-safe & Right-biased only \\
\hline
Validated & Error accumulation & Collects all errors & More complex \\
\hline
Union Types & Scala 3 errors & Modern, efficient & New syntax \\
\hline
IO/Effect & Async + Sync & Composable & Learning curve \\
\hline
\end{tabular}
\end{table}

\begin{itemize}
\item Choose based on your specific requirements
\item Migrate gradually from exceptions to functional types
\item Consider performance implications
\end{itemize}

\end{frame}

%---------------------------------------------------------
\begin{frame}[fragile]
\frametitle{Key Takeaways}

\begin{enumerate}
\item \textbf{Make errors explicit} in function signatures
\item \textbf{Use Option} for simple null/missing value cases
\item \textbf{Use Either} when you need error information
\item \textbf{Use Validated} when you need to accumulate errors
\item \textbf{Consider Union types} in Scala 3 for performance
\item \textbf{Design error hierarchies} with sealed traits
\item \textbf{Test error scenarios} thoroughly
\item \textbf{Provide recovery mechanisms} where appropriate
\end{enumerate}

\vspace{1em}
\textit{Functional error handling leads to more robust, composable, and maintainable code.}

\end{frame}

%---------------------------------------------------------
\begin{frame}{References and Further Reading}
\footnotesize

\begin{thebibliography}{9}
\beamertemplatetextbibitems

\subsection*{Scala Error Handling}
\bibitem[Odersky, 2021]{odersky2021}
\textcolor{pureblue}{Odersky, M.} 
\textit{Scala 3 Reference: Union Types}. 2021.
\footnotesize{\\ \color{gray}Official documentation on union types for error handling}

\bibitem[Spiewak, 2018]{spiewak2018}
\textcolor{pureblue}{Spiewak, D.} 
\textit{Functional Error Handling with Monads}. 2018.
\footnotesize{\\ \color{gray}Comprehensive guide to monadic error handling}

\subsection*{Functional Programming}
\bibitem[Lipovaca, 2011]{lipovaca2011}
\textcolor{pureblue}{Lipovaca, M.} 
\textit{Learn You a Haskell: Error Handling}. 2011.
\footnotesize{\\ \color{gray}Fundamental concepts of functional error handling}

\bibitem[Cats Contributors, 2023]{cats2023}
\textcolor{pureblue}{Cats Contributors} 
\textit{Cats Documentation: Error Handling}. 2023.
\footnotesize{\\ \color{gray}Practical patterns with cats library}

\subsection*{Best Practices}
\bibitem[Wampler, 2020]{wampler2020}
\textcolor{pureblue}{Wampler, D.} 
\textit{Programming Scala: Error Handling}. 2020.
\footnotesize{\\ \color{gray}Industry best practices for Scala error handling}

\end{thebibliography}

\end{frame}

\end{document}
