\documentclass{beamer}
\usepackage[utf8]{inputenc}
\usepackage{booktabs}
\usepackage{qrcode}
\usepackage{graphicx}
\usepackage{listings}
\usepackage{xcolor}
\usepackage{multirow}
\usepackage{array}

% Define Scala syntax highlighting
\lstdefinestyle{scalaStyle}{
  language=Scala,
  basicstyle=\ttfamily\small,
  keywordstyle=\color{blue},
  commentstyle=\color{brown},
  stringstyle=\color{red},
  showstringspaces=false,
  breaklines=true,
  frame=single,
  numbers=left,
  numberstyle=\tiny\color{gray}
}

\definecolor{scalared}{RGB}{215,50,58}
\definecolor{darkblue}{RGB}{0,0,139}
\definecolor{pureblue}{RGB}{0,82,155}
\definecolor{impurered}{RGB}{186,0,13}
\definecolor{titleblue}{RGB}{0,82,155}
\definecolor{alertred}{RGB}{186,0,13}
\definecolor{darkgray}{RGB}{80,80,80}

\usetheme{Madrid}
\usecolortheme{default}

\title[Error Handling in Scala]
{Functional programming with Scala}

\subtitle{Error Handling}

\author[Said BOUDJELDA]
{Said BOUDJELDA}

\institute[efrei]
{
  Senior Software Engineer @SCIAM\\
  GitHub @bmscomp
}

\date[efrei 2025]
{Course, May 2025}

\logo{\includegraphics[height=0.6cm]{logo}}

\begin{document}

\frame{\titlepage}

%---------------------------------------------------------
\begin{frame}
\frametitle{Overview}

Modern error handling in Scala 3:

\begin{itemize}
\item \textbf{Union Types}: Native error representation
\item \textbf{Enums}: Structured error hierarchies
\item \textbf{Type Safety}: Errors visible in signatures
\item \textbf{Zero Cost}: No performance overhead
\item \textbf{Composition}: Functional error chaining
\end{itemize}

\textit{From exceptions to types}

\end{frame}

%---------------------------------------------------------
\begin{frame}[fragile]
\frametitle{Traditional vs Modern}

\begin{lstlisting}[style=scalaStyle]
// Old way: Hidden exceptions
def divide(a: Int, b: Int): Int = 
  if b == 0 then throw ArithmeticException("Zero")
  else a / b

// Scala 3 way: Union types
def safeDivide(a: Int, b: Int): Int | String = 
  if b == 0 then "Division by zero" else a / b
\end{lstlisting}

\textbf{Modern approach}: Type-safe, no hidden failures

\end{frame}

%---------------------------------------------------------
\begin{frame}
\frametitle{Problems with Exception-Based Error Handling}

\begin{itemize}
  \item \textbf{Hidden Failures}: No compile-time visibility of errors
  \item \textbf{Type System Bypass}: Exceptions break normal type flow
  \item \textbf{Performance Overhead}: Stack unwinding is expensive
  \item \textbf{Resource Leaks}: Finally blocks can be error-prone
  \item \textbf{Poor Composability}: Hard to chain operations safely
  \item \textbf{Testing Difficulties}: Exception paths often forgotten
  \item \textbf{Maintenance Issues}: Catching wrong exception types
\end{itemize}

\vspace{0.3cm}
\textit{Modern functional approaches solve these problems}

\end{frame}

%---------------------------------------------------------
\begin{frame}[fragile]
\frametitle{Union Types}

\begin{lstlisting}[style=scalaStyle]
type Result[T] = T | String

def parse(s: String): Result[Int] = 
  try s.toInt catch case _ => "Invalid number"

parse("42") match
  case num: Int => println(s"Success: $num")
  case err: String => println(s"Error: $err")
\end{lstlisting}

Simple, type-safe error handling

\end{frame}

%---------------------------------------------------------
\begin{frame}[fragile]
\frametitle{Enums for Error Types}

\begin{lstlisting}[style=scalaStyle]
enum AppError:
  case ValidationError(msg: String)
  case NetworkError(msg: String)
  case ParseError(msg: String)

def validateAge(age: Int): Int | AppError =
  if age > 0 then age 
  else AppError.ValidationError("Invalid age")
\end{lstlisting}

Structured error hierarchies

\end{frame}

%---------------------------------------------------------
\begin{frame}[fragile]
\frametitle{Extension Methods}

\begin{lstlisting}[style=scalaStyle]
extension [T](value: T)
  def ensure(condition: T => Boolean, 
             error: String): T | String =
    if condition(value) then value else error

// Usage
"hello".ensure(_.nonEmpty, "Empty string")
42.ensure(_ > 0, "Must be positive")
\end{lstlisting}

Fluent validation APIs

\end{frame}

%---------------------------------------------------------
\begin{frame}[fragile]
\frametitle{Chaining Operations}

\begin{lstlisting}[style=scalaStyle]
extension [T](result: T | String)
  def andThen[U](f: T => U | String): U | String = 
    result match
      case value: T => f(value)
      case error: String => error

"42".ensure(_.nonEmpty, "Empty")
  .andThen(s => parse(s))
  .andThen(n => n.ensure(_ > 0, "Negative"))
\end{lstlisting}

Composable error handling

\end{frame}

%---------------------------------------------------------
\begin{frame}[fragile]
\frametitle{Option for Null Safety}

\begin{lstlisting}[style=scalaStyle]
def findUser(id: Int): Option[String] = 
  if id > 0 then Some(s"User$id") else None

val result = findUser(1)
  .map(_.toUpperCase)
  .getOrElse("Not found")

println(result) // USER1
\end{lstlisting}

Clean null handling

\end{frame}

%---------------------------------------------------------
\begin{frame}[fragile]
\frametitle{Either for Rich Errors}

\begin{lstlisting}[style=scalaStyle]
def validateUser(name: String, age: Int): Either[String, String] = 
  for
    n <- if name.nonEmpty then Right(name) else Left("No name")
    a <- if age > 0 then Right(age) else Left("Bad age")
  yield s"$n is $a years old"

println(validateUser("John", 25))
// Right(John is 25 years old)
\end{lstlisting}

For-comprehension error handling

\end{frame}

%---------------------------------------------------------
\begin{frame}[fragile]
\frametitle{Resource Management}

\begin{lstlisting}[style=scalaStyle]
import scala.util.Using

def readFile(name: String): String | String = 
  Using(scala.io.Source.fromFile(name)) { source =>
    source.mkString
  }.toEither match
    case Right(content) => content
    case Left(ex) => s"Error: ${ex.getMessage}"
\end{lstlisting}

Automatic cleanup

\end{frame}

%---------------------------------------------------------
\begin{frame}[fragile]
\frametitle{Given/Using Context}

\begin{lstlisting}[style=scalaStyle]
trait Logger:
  def log(msg: String): Unit

given Logger with
  def log(msg: String) = println(s"[LOG] $msg")

def validate[T](value: T)(using logger: Logger): Option[T] =
  logger.log(s"Validating $value")
  Some(value)
\end{lstlisting}

Contextual error handling

\end{frame}

%---------------------------------------------------------
\begin{frame}[fragile]
\frametitle{Async Error Handling}

\begin{lstlisting}[style=scalaStyle]
import scala.concurrent.Future

def fetchUser(id: Int): Future[String | AppError] = 
  Future {
    if id > 0 then s"User$id"
    else AppError.ValidationError("Invalid ID")
  }

fetchUser(1).map {
  case user: String => s"Found: $user"
  case error: AppError => s"Error: $error"
}
\end{lstlisting}

Future with union types

\end{frame}

%---------------------------------------------------------
\begin{frame}[fragile]
\frametitle{Pattern Matching}

\begin{lstlisting}[style=scalaStyle]
def handle(result: String | Int | AppError): String = result match
  case s: String => s"Text: $s"
  case n: Int => s"Number: $n"
  case AppError.ValidationError(msg) => s"Invalid: $msg"
  case AppError.NetworkError(msg) => s"Network: $msg"
  case AppError.ParseError(msg) => s"Parse: $msg"
\end{lstlisting}

Exhaustive error handling

\end{frame}

%---------------------------------------------------------
\begin{frame}[fragile]
\frametitle{Error Boundaries}

\begin{lstlisting}[style=scalaStyle]
def safe[T](operation: () => T): T | AppError =
  try operation()
  catch 
    case _: NumberFormatException => AppError.ParseError("Bad format")
    case ex: Exception => AppError.NetworkError(ex.getMessage)

val result = safe(() => "42".toInt)
// Success: 42
\end{lstlisting}

Isolate failures

\end{frame}

%---------------------------------------------------------
\begin{frame}[fragile]
\frametitle{Validation DSL}

\begin{lstlisting}[style=scalaStyle]
case class Validated[T](value: T, errors: List[String])

extension [T](value: T)
  def validate: Validated[T] = Validated(value, Nil)

extension [T](v: Validated[T])
  def check(cond: T => Boolean, err: String): Validated[T] =
    if cond(v.value) then v else v.copy(errors = v.errors :+ err)
\end{lstlisting}

Custom validation builder

\end{frame}

%---------------------------------------------------------
\begin{frame}[fragile]
\frametitle{Testing Errors}

\begin{lstlisting}[style=scalaStyle]
def testValidation(): Unit =
  // Test success
  assert(validateAge(25) == 25)
  
  // Test error
  validateAge(-5) match
    case AppError.ValidationError(msg) => assert(msg.contains("Invalid"))
    case _ => sys.error("Expected error")
\end{lstlisting}

Simple error testing

\end{frame}

%---------------------------------------------------------
\begin{frame}[fragile]
\frametitle{Performance: Zero Cost}

\begin{lstlisting}[style=scalaStyle]
// Union types - no boxing overhead
type FastResult = String | Int

def quickParse(s: String): FastResult =
  if s.forall(_.isDigit) then s.toInt else "Invalid"

// Opaque types for domain safety
opaque type UserId = Int
object UserId:
  def apply(id: Int): UserId | String = 
    if id > 0 then id else "Invalid ID"
\end{lstlisting}

Efficient error handling

\end{frame}

%---------------------------------------------------------
\begin{frame}[fragile]
\frametitle{Real-World API}

\begin{lstlisting}[style=scalaStyle]
case class User(name: String, age: Int)

def createUser(name: String, age: String): User | List[String] =
  val validations = List(
    if name.nonEmpty then None else Some("Name required"),
    try { age.toInt; None } catch case _ => Some("Invalid age")
  ).flatten
  
  if validations.isEmpty then User(name, age.toInt) else validations
\end{lstlisting}

Complete validation example

\end{frame}

%---------------------------------------------------------
\begin{frame}[fragile]
\frametitle{Migration Strategy}

\begin{lstlisting}[style=scalaStyle]
// Step 1: Wrap exceptions
def legacy(): String = throw RuntimeException("Old")
def wrapped(): String | String = 
  try legacy() catch case ex => ex.getMessage

// Step 2: Use union types
def modern(): String | AppError =
  AppError.NetworkError("Converted")
\end{lstlisting}

Gradual migration path

\end{frame}

%---------------------------------------------------------
\begin{frame}[fragile]
\frametitle{Logging and Monitoring}

\begin{lstlisting}[style=scalaStyle]
def withLogging[T](op: String)(f: => T | AppError): T | AppError =
  val start = System.currentTimeMillis()
  f match
    case result: T => 
      println(s"$op: success (${System.currentTimeMillis() - start}ms)")
      result
    case error: AppError => 
      println(s"$op: error - ${error}")
      error
\end{lstlisting}

Structured error tracking

\end{frame}

%---------------------------------------------------------
\begin{frame}[fragile]
\frametitle{Comparison Table}

\begin{table}[ht]
\centering
\small
\begin{tabular}{|l|l|l|}
\hline
\textbf{Pattern} & \textbf{Use Case} & \textbf{Performance} \\
\hline
Union Types & Simple errors & Zero cost \\
\hline
Enums & Structured errors & Zero cost \\
\hline
Option & Null safety & Minimal overhead \\
\hline
Either & Rich errors & Some allocation \\
\hline
Try & Exception wrapping & Exception cost \\
\hline
\end{tabular}
\end{table}

\textbf{Recommendation:} Union types for most cases

\end{frame}

%---------------------------------------------------------
\begin{frame}[fragile]
\frametitle{Best Practices}

\begin{enumerate}
\item \textbf{Union types} for simple cases
\item \textbf{Enums} for error hierarchies  
\item \textbf{Extensions} for fluent APIs
\item \textbf{Pattern matching} over try-catch
\item \textbf{Compose} operations safely
\item \textbf{Test} error scenarios
\end{enumerate}

\begin{lstlisting}[style=scalaStyle]
// Good: Explicit errors
def parse(s: String): Int | String

// Bad: Hidden exceptions  
def parse(s: String): Int
\end{lstlisting}

\end{frame}

%---------------------------------------------------------
\begin{frame}[fragile]
\frametitle{Advanced: Custom Error Monad}

\begin{lstlisting}[style=scalaStyle]
case class Result[T](value: T | String):
  def map[U](f: T => U): Result[U] = value match
    case v: T => Result(f(v))
    case e: String => Result(e)
  
  def flatMap[U](f: T => Result[U]): Result[U] = value match
    case v: T => f(v)
    case e: String => Result(e)
\end{lstlisting}

Build your own error monad

\end{frame}

%---------------------------------------------------------
\begin{frame}[fragile]
\frametitle{HTTP Client Example}

\begin{lstlisting}[style=scalaStyle]
enum HttpError:
  case NotFound, Unauthorized, ServerError

def get(url: String): String | HttpError =
  if url.startsWith("https://") then "Response data"
  else HttpError.NotFound

get("https://api.example.com") match
  case data: String => println(s"Success: $data")
  case HttpError.NotFound => println("URL not found")
\end{lstlisting}

Simple HTTP error handling

\end{frame}

%---------------------------------------------------------
\begin{frame}[fragile]
\frametitle{Database Operations}

\begin{lstlisting}[style=scalaStyle]
enum DbError:
  case ConnectionFailed, QueryFailed, NotFound

def findById(id: Int): User | DbError =
  if id > 0 then User("John", 25)
  else DbError.NotFound

def updateUser(user: User): Unit | DbError =
  if user.name.nonEmpty then () else DbError.QueryFailed
\end{lstlisting}

Database error modeling

\end{frame}

%---------------------------------------------------------
\begin{frame}[fragile]
\frametitle{Validation Pipeline}

\begin{lstlisting}[style=scalaStyle]
def validateEmail(email: String): String | String =
  email.ensure(_.nonEmpty, "Email required")
    .andThen(_.ensure(_.contains("@"), "Invalid format"))
    .andThen(_.ensure(_.length < 100, "Too long"))

def processSignup(email: String) = validateEmail(email) match
  case valid: String => s"Welcome: $valid"
  case error: String => s"Error: $error"
\end{lstlisting}

Composable validation chain

\end{frame}

%---------------------------------------------------------
\begin{frame}[fragile]
\frametitle{Configuration Loading}

\begin{lstlisting}[style=scalaStyle]
case class Config(host: String, port: Int)

def loadConfig(): Config | String =
  for
    host <- sys.env.get("HOST").toRight("Missing HOST")
    portStr <- sys.env.get("PORT").toRight("Missing PORT")  
    port <- try Right(portStr.toInt) catch case _ => Left("Invalid PORT")
  yield Config(host, port)
\end{lstlisting}

Environment configuration

\end{frame}

%---------------------------------------------------------
\begin{frame}[fragile]
\frametitle{JSON Parsing}

\begin{lstlisting}[style=scalaStyle]
def parseJson(json: String): Map[String, String] | String =
  try 
    val data = json.split(",").map(_.split(":")).map { arr =>
      arr(0).trim -> arr(1).trim
    }.toMap
    data
  catch case _ => "Invalid JSON format"

parseJson("name:John,age:25") match
  case data: Map[String, String] => println(data)
  case error: String => println(s"Parse error: $error")
\end{lstlisting}

Simple JSON parsing with errors

\end{frame}

%---------------------------------------------------------
\begin{frame}[fragile]
\frametitle{Key Takeaways}

\begin{itemize}
\item \textbf{Union types} are the modern Scala 3 way
\item \textbf{Zero cost} error handling
\item \textbf{Type safety} prevents surprises
\item \textbf{Pattern matching} handles all cases
\item \textbf{Composition} builds complex validations
\item \textbf{Migration} can be gradual
\end{itemize}

\vspace{1em}
\textit{Make errors visible in types, not hidden in exceptions}

\end{frame}

%---------------------------------------------------------
\begin{frame}{References}
\footnotesize

\begin{thebibliography}{4}
\beamertemplatetextbibitems

\bibitem[Odersky, 2023]{odersky2023}
\textcolor{pureblue}{Odersky, M.} 
\textit{Scala 3 Reference: Union Types}. 2023.

\bibitem[EPFL, 2023]{epfl2023}
\textcolor{pureblue}{EPFL Team} 
\textit{Scala 3 Book: Error Handling}. 2023.

\bibitem[Wampler, 2021]{wampler2021}
\textcolor{pureblue}{Wampler, D.} 
\textit{Programming Scala 3}. 2021.

\bibitem[Spiewak, 2020]{spiewak2020}
\textcolor{pureblue}{Spiewak, D.} 
\textit{Functional Error Handling}. 2020.

\end{thebibliography}

\end{frame}

\end{document}
