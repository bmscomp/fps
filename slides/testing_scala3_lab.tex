\documentclass[12pt,a4paper]{article}
\usepackage[utf8]{inputenc}
\usepackage[T1]{fontenc}
\usepackage{amsmath}
\usepackage{amsfonts}
\usepackage{amssymb}
\usepackage{listings}
\usepackage{xcolor}
\usepackage{geometry}
\usepackage{fancyhdr}
\usepackage{titlesec}
\usepackage{graphicx}
\usepackage{hyperref}
\usepackage{booktabs}
\usepackage{tabularx}

\geometry{margin=2.5cm}

\titleformat{\section}{\normalfont\Large\bfseries}{Lab \thesection:}{1em}{}
\lstset{
  basicstyle=\ttfamily\small,
  keywordstyle=\color{blue},
  commentstyle=\color{gray},
  stringstyle=\color{orange},
  showstringspaces=false,
  breaklines=true,
  frame=single,
  tabsize=2,
  language=Scala
}

\title{\textbf{Unit Testing in Scala 3 - Practical Lab Session}}
\author{\textbf{Said BOUDJELDA}\\ \textbf{Paris-Panthéon-Assas University}}
\date{\today}

\begin{document}

\maketitle

\begin{center}
  \includegraphics[width=5cm]{logo}
\end{center}

\tableofcontents
\newpage

\section{Lab Overview}

\subsection{Learning Objectives}
By the end of this lab session, students will be able to:
\begin{itemize}
  \item Set up a Scala 3 project with testing dependencies
  \item Write effective unit tests using ScalaTest framework
  \item Apply Test-Driven Development (TDD) methodology
  \item Implement property-based testing with ScalaCheck
  \item Use mocking frameworks for testing dependencies
  \item Understand testing best practices and patterns
  \item Measure and improve test coverage
\end{itemize}

\subsection{Prerequisites}
\begin{itemize}
  \item Basic knowledge of Scala 3 syntax and concepts
  \item Understanding of object-oriented and functional programming
  \item Familiarity with SBT build tool
  \item IDE setup (IntelliJ IDEA or VS Code with Metals)
\end{itemize}

\subsection{Duration and Structure}
\begin{table}[h]
\centering
\begin{tabularx}{\textwidth}{|l|X|r|}
\hline
\textbf{Activity} & \textbf{Description} & \textbf{Time} \\
\hline
Setup & Project setup and dependencies & 15 min \\
Part 1 & Basic unit testing with ScalaTest & 45 min \\
Break & Coffee break & 15 min \\
Part 2 & TDD implementation & 45 min \\
Part 3 & Property-based testing & 30 min \\
Part 4 & Mocking and advanced patterns & 45 min \\
Wrap-up & Q\&A and assessment & 15 min \\
\hline
\textbf{Total} & & \textbf{3h 30min} \\
\hline
\end{tabularx}
\end{table}

\section{Materials and Setup}

\subsection{Required Software}
\begin{itemize}
  \item \textbf{Java JDK 11+}: OpenJDK or Oracle JDK
  \item \textbf{SBT 1.8+}: Scala Build Tool
  \item \textbf{IDE}: IntelliJ IDEA Community with Scala plugin OR VS Code with Metals
  \item \textbf{Git}: For version control
\end{itemize}

\subsection{Project Dependencies}
Create a new SBT project with the following \texttt{build.sbt} configuration:

\begin{lstlisting}
ThisBuild / version := "0.1.0-SNAPSHOT"
ThisBuild / scalaVersion := "3.3.0"

lazy val root = (project in file("."))
  .settings(
    name := "scala3-testing-lab",
    libraryDependencies ++= Seq(
      // Testing frameworks
      "org.scalatest" %% "scalatest" % "3.2.15" % Test,
      "org.scalatestplus" %% "scalacheck-1-17" % "3.2.15.0" % Test,
      "org.scalacheck" %% "scalacheck" % "1.17.0" % Test,
      
      // Mocking
      "org.mockito" %% "mockito-scala" % "1.17.12" % Test,
      "org.scalatestplus" %% "mockito-4-6" % "3.2.15.0" % Test,
      
      // Production dependencies
      "org.typelevel" %% "cats-core" % "2.9.0",
      "org.typelevel" %% "cats-effect" % "3.4.8"
    ),
    
    // Test configuration
    Test / parallelExecution := false,
    Test / testOptions += Tests.Argument("-oDF")
  )
\end{lstlisting}

\subsection{Project Structure}
\begin{lstlisting}
scala3-testing-lab/
├── build.sbt
├── project/
│   ├── build.properties
│   └── plugins.sbt
├── src/
│   ├── main/scala/
│   │   ├── domain/
│   │   ├── service/
│   │   └── utils/
│   └── test/scala/
│       ├── domain/
│       ├── service/
│       └── utils/
└── README.md
\end{lstlisting}

\section{Part 1: Basic Unit Testing (45 minutes)}

\subsection{Activity 1.1: Your First Test Class (15 min)}

\textbf{Objective:} Create a simple calculator and write basic unit tests.

\textbf{Step 1:} Create the Calculator class in \texttt{src/main/scala/utils/Calculator.scala}:

\begin{lstlisting}
package utils

object Calculator:
  def add(a: Int, b: Int): Int = a + b
  def subtract(a: Int, b: Int): Int = a - b
  def multiply(a: Int, b: Int): Int = a * b
  def divide(a: Int, b: Int): Double = 
    if b == 0 then throw new ArithmeticException("Division by zero")
    else a.toDouble / b
\end{lstlisting}

\textbf{Step 2:} Create test class in \texttt{src/test/scala/utils/CalculatorSpec.scala}:

\begin{lstlisting}
package utils

import org.scalatest.flatspec.AnyFlatSpec
import org.scalatest.matchers.should.Matchers

class CalculatorSpec extends AnyFlatSpec with Matchers:
  
  "Calculator" should "add two numbers correctly" in {
    Calculator.add(2, 3) should be(5)
    Calculator.add(-1, 1) should be(0)
    Calculator.add(0, 0) should be(0)
  }
  
  it should "subtract two numbers correctly" in {
    Calculator.subtract(5, 3) should be(2)
    Calculator.subtract(1, 1) should be(0)
    Calculator.subtract(0, 5) should be(-5)
  }
  
  it should "multiply two numbers correctly" in {
    Calculator.multiply(3, 4) should be(12)
    Calculator.multiply(-2, 3) should be(-6)
    Calculator.multiply(0, 10) should be(0)
  }
  
  it should "divide two numbers correctly" in {
    Calculator.divide(10, 2) should be(5.0)
    Calculator.divide(7, 2) should be(3.5)
  }
  
  it should "throw exception when dividing by zero" in {
    assertThrows[ArithmeticException] {
      Calculator.divide(10, 0)
    }
  }
\end{lstlisting}

\textbf{Step 3:} Run the tests using: \texttt{sbt test}

\subsection{Activity 1.2: Testing Collections (15 min)}

\textbf{Objective:} Practice testing functions that work with collections.

Create \texttt{src/main/scala/utils/ListUtils.scala}:

\begin{lstlisting}
package utils

object ListUtils:
  def findMax(numbers: List[Int]): Option[Int] = 
    if numbers.isEmpty then None
    else Some(numbers.max)
  
  def filterEven(numbers: List[Int]): List[Int] = 
    numbers.filter(_ % 2 == 0)
  
  def groupByLength(words: List[String]): Map[Int, List[String]] = 
    words.groupBy(_.length)
\end{lstlisting}

Create corresponding test in \texttt{src/test/scala/utils/ListUtilsSpec.scala}:

\begin{lstlisting}
package utils

import org.scalatest.flatspec.AnyFlatSpec
import org.scalatest.matchers.should.Matchers

class ListUtilsSpec extends AnyFlatSpec with Matchers:
  
  "ListUtils.findMax" should "return the maximum value" in {
    ListUtils.findMax(List(1, 5, 3, 9, 2)) should be(Some(9))
    ListUtils.findMax(List(-1, -5, -2)) should be(Some(-1))
  }
  
  it should "return None for empty list" in {
    ListUtils.findMax(List.empty) should be(None)
  }
  
  "ListUtils.filterEven" should "return only even numbers" in {
    ListUtils.filterEven(List(1, 2, 3, 4, 5, 6)) should be(List(2, 4, 6))
    ListUtils.filterEven(List(1, 3, 5)) should be(List.empty)
    ListUtils.filterEven(List.empty) should be(List.empty)
  }
  
  "ListUtils.groupByLength" should "group words by their length" in {
    val words = List("cat", "dog", "elephant", "ox")
    val result = ListUtils.groupByLength(words)
    
    result should have size 3
    result(2) should contain only "ox"
    result(3) should contain allOf("cat", "dog")
    result(8) should contain only "elephant"
  }
\end{lstlisting}

\subsection{Activity 1.3: Testing with Custom Matchers (15 min)}

\textbf{Objective:} Create domain-specific test matchers.

Create \texttt{src/main/scala/domain/Person.scala}:

\begin{lstlisting}
package domain

case class Person(name: String, age: Int, email: String):
  def isAdult: Boolean = age >= 18
  def hasValidEmail: Boolean = email.contains("@") && email.contains(".")
\end{lstlisting}

Create test with custom matchers in \texttt{src/test/scala/domain/PersonSpec.scala}:

\begin{lstlisting}
package domain

import org.scalatest.flatspec.AnyFlatSpec
import org.scalatest.matchers.{BeMatcher, MatchResult, Matcher, should}

class PersonSpec extends AnyFlatSpec with should.Matchers:
  
  // Custom matchers
  val beAdult: BeMatcher[Person] = BeMatcher { person =>
    MatchResult(
      person.isAdult,
      s"${person.name} was not an adult (age: ${person.age})",
      s"${person.name} was an adult (age: ${person.age})"
    )
  }
  
  def haveValidEmail: Matcher[Person] = Matcher { person =>
    MatchResult(
      person.hasValidEmail,
      s"${person.email} was not a valid email",
      s"${person.email} was a valid email"
    )
  }
  
  "Person" should "be identified as adult correctly" in {
    val adult = Person("Alice", 25, "alice@example.com")
    val minor = Person("Bob", 16, "bob@example.com")
    
    adult should beAdult
    minor should not be beAdult
  }
  
  it should "validate email addresses" in {
    val validPerson = Person("Alice", 25, "alice@example.com")
    val invalidPerson = Person("Bob", 25, "invalid-email")
    
    validPerson should haveValidEmail
    invalidPerson should not(haveValidEmail)
  }
\end{lstlisting}

\section{Part 2: Test-Driven Development (45 minutes)}

\subsection{Activity 2.1: TDD Kata - String Calculator (45 min)}

\textbf{Objective:} Practice TDD by implementing a string calculator step by step.

\textbf{Requirements:} Implement a string calculator with these requirements:
\begin{enumerate}
  \item Handle empty string (return 0)
  \item Handle single number (return the number)
  \item Handle two numbers separated by comma (return sum)
  \item Handle multiple numbers (return sum)
  \item Handle new line delimiter
  \item Handle custom delimiter
  \item Throw exception for negative numbers
\end{enumerate}

\textbf{Step 1:} Create the failing test first in \texttt{src/test/scala/utils/StringCalculatorSpec.scala}:

\begin{lstlisting}
package utils

import org.scalatest.flatspec.AnyFlatSpec
import org.scalatest.matchers.should.Matchers

class StringCalculatorSpec extends AnyFlatSpec with Matchers:
  
  val calculator = new StringCalculator()
  
  // Requirement 1: Empty string
  "StringCalculator" should "return 0 for empty string" in {
    calculator.add("") should be(0)
  }
  
  // Requirement 2: Single number
  it should "return the number for single digit" in {
    calculator.add("1") should be(1)
    calculator.add("5") should be(5)
  }
  
  // Continue with more tests...
\end{lstlisting}

\textbf{Step 2:} Create minimal implementation in \texttt{src/main/scala/utils/StringCalculator.scala}:

\begin{lstlisting}
package utils

class StringCalculator:
  def add(numbers: String): Int = 
    if numbers.isEmpty then 0
    else numbers.toInt
\end{lstlisting}

\textbf{Step 3:} Run test, see it pass, then add next test and implement incrementally.

\textbf{Complete implementation after all TDD cycles:}

\begin{lstlisting}
package utils

class StringCalculator:
  def add(numbers: String): Int = 
    if numbers.isEmpty then 0
    else 
      val delimiter = extractDelimiter(numbers)
      val cleanNumbers = extractNumbers(numbers)
      val nums = parseNumbers(cleanNumbers, delimiter)
      
      val negatives = nums.filter(_ < 0)
      if negatives.nonEmpty then
        throw new IllegalArgumentException(s"Negatives not allowed: ${negatives.mkString(",")}")
      
      nums.sum
  
  private def extractDelimiter(numbers: String): String = 
    if numbers.startsWith("//") then
      numbers.charAt(2).toString
    else "[,\n]"
  
  private def extractNumbers(numbers: String): String = 
    if numbers.startsWith("//") then
      numbers.substring(4)
    else numbers
  
  private def parseNumbers(numbers: String, delimiter: String): List[Int] = 
    if numbers.isEmpty then List(0)
    else numbers.split(delimiter).map(_.toInt).toList
\end{lstlisting}

\section{Part 3: Property-Based Testing (30 minutes)}

\subsection{Activity 3.1: ScalaCheck Properties (30 min)}

\textbf{Objective:} Write property-based tests for mathematical operations.

Create \texttt{src/test/scala/utils/MathPropertiesSpec.scala}:

\begin{lstlisting}
package utils

import org.scalatest.flatspec.AnyFlatSpec
import org.scalatest.matchers.should.Matchers
import org.scalatestplus.scalacheck.ScalaCheckPropertyChecks

class MathPropertiesSpec extends AnyFlatSpec 
  with Matchers 
  with ScalaCheckPropertyChecks:
  
  "String reverse" should "satisfy double reversal property" in {
    forAll { (s: String) =>
      val reversed = s.reverse
      reversed.reverse should be(s)
    }
  }
  
  "List concatenation" should "be associative" in {
    forAll { (a: List[Int], b: List[Int], c: List[Int]) =>
      (a ++ b) ++ c should be(a ++ (b ++ c))
    }
  }
  
  "Addition" should "be commutative" in {
    forAll { (a: Int, b: Int) =>
      whenever(math.abs(a.toLong + b.toLong) <= Int.MaxValue) {
        Calculator.add(a, b) should be(Calculator.add(b, a))
      }
    }
  }
  
  "Multiplication by zero" should "always return zero" in {
    forAll { (n: Int) =>
      Calculator.multiply(n, 0) should be(0)
      Calculator.multiply(0, n) should be(0)
    }
  }
\end{lstlisting}

\section{Part 4: Mocking and Advanced Patterns (45 minutes)}

\subsection{Activity 4.1: Testing with Dependencies (25 min)}

\textbf{Objective:} Learn to test classes with external dependencies using mocks.

Create \texttt{src/main/scala/service/UserService.scala}:

\begin{lstlisting}
package service

import domain.Person

trait UserRepository:
  def findById(id: Long): Option[Person]
  def save(person: Person): Person
  def delete(id: Long): Boolean

class UserService(repository: UserRepository):
  def getUser(id: Long): Option[Person] = 
    repository.findById(id)
  
  def createUser(name: String, age: Int, email: String): Person = 
    val person = Person(name, age, email)
    if !person.hasValidEmail then
      throw new IllegalArgumentException("Invalid email")
    repository.save(person)
  
  def deleteUser(id: Long): Boolean = 
    repository.findById(id) match
      case Some(_) => repository.delete(id)
      case None => false
\end{lstlisting}

Create test with mocks in \texttt{src/test/scala/service/UserServiceSpec.scala}:

\begin{lstlisting}
package service

import domain.Person
import org.mockito.MockitoSugar
import org.scalatest.flatspec.AnyFlatSpec
import org.scalatest.matchers.should.Matchers

class UserServiceSpec extends AnyFlatSpec with Matchers with MockitoSugar:
  
  "UserService.getUser" should "return user from repository" in {
    val mockRepo = mock[UserRepository]
    val service = new UserService(mockRepo)
    val user = Person("Alice", 25, "alice@test.com")
    
    when(mockRepo.findById(1L)).thenReturn(Some(user))
    
    service.getUser(1L) should be(Some(user))
    verify(mockRepo).findById(1L)
  }
  
  "UserService.createUser" should "save valid user" in {
    val mockRepo = mock[UserRepository]
    val service = new UserService(mockRepo)
    val user = Person("Bob", 30, "bob@test.com")
    
    when(mockRepo.save(user)).thenReturn(user)
    
    service.createUser("Bob", 30, "bob@test.com") should be(user)
    verify(mockRepo).save(user)
  }
  
  it should "throw exception for invalid email" in {
    val mockRepo = mock[UserRepository]
    val service = new UserService(mockRepo)
    
    assertThrows[IllegalArgumentException] {
      service.createUser("Invalid", 25, "invalid-email")
    }
    
    verify(mockRepo, never).save(any[Person])
  }
  
  "UserService.deleteUser" should "delete existing user" in {
    val mockRepo = mock[UserRepository]
    val service = new UserService(mockRepo)
    val user = Person("Charlie", 35, "charlie@test.com")
    
    when(mockRepo.findById(1L)).thenReturn(Some(user))
    when(mockRepo.delete(1L)).thenReturn(true)
    
    service.deleteUser(1L) should be(true)
    verify(mockRepo).findById(1L)
    verify(mockRepo).delete(1L)
  }
  
  it should "return false for non-existing user" in {
    val mockRepo = mock[UserRepository]
    val service = new UserService(mockRepo)
    
    when(mockRepo.findById(1L)).thenReturn(None)
    
    service.deleteUser(1L) should be(false)
    verify(mockRepo).findById(1L)
    verify(mockRepo, never).delete(any[Long])
  }
\end{lstlisting}

\subsection{Activity 4.2: Test Data Builders (20 min)}

\textbf{Objective:} Implement the test data builder pattern for complex objects.

Create \texttt{src/test/scala/utils/PersonBuilder.scala}:

\begin{lstlisting}
package utils

import domain.Person

case class PersonBuilder private(
  name: String = "Default Name",
  age: Int = 25,
  email: String = "default@test.com"
):
  def withName(name: String): PersonBuilder = copy(name = name)
  def withAge(age: Int): PersonBuilder = copy(age = age)
  def withEmail(email: String): PersonBuilder = copy(email = email)
  def adult: PersonBuilder = copy(age = 25)
  def minor: PersonBuilder = copy(age = 16)
  def invalidEmail: PersonBuilder = copy(email = "invalid-email")
  
  def build(): Person = Person(name, age, email)

object PersonBuilder:
  def apply(): PersonBuilder = new PersonBuilder()
\end{lstlisting}

Use in tests:

\begin{lstlisting}
"PersonBuilder" should "create test data easily" in {
  val adult = PersonBuilder()
    .withName("Alice")
    .withAge(30)
    .withEmail("alice@example.com")
    .build()
  
  val minor = PersonBuilder()
    .withName("Bob")
    .minor
    .build()
  
  adult.isAdult should be(true)
  minor.isAdult should be(false)
}
\end{lstlisting}

\section{Assessment and Deliverables}

\subsection{Lab Assessment Criteria}
\begin{table}[h]
\centering
\begin{tabularx}{\textwidth}{|l|X|r|}
\hline
\textbf{Criteria} & \textbf{Description} & \textbf{Points} \\
\hline
Test Coverage & All methods have corresponding tests & 25\% \\
Test Quality & Tests are clear, focused, and follow best practices & 25\% \\
TDD Implementation & Proper red-green-refactor cycle demonstrated & 20\% \\
Advanced Techniques & Use of property-based testing, mocking, builders & 20\% \\
Code Organization & Clean structure and readable code & 10\% \\
\hline
\textbf{Total} & & \textbf{100\%} \\
\hline
\end{tabularx}
\end{table}

\subsection{Deliverables}
Students should submit:
\begin{enumerate}
  \item Complete SBT project with all implemented classes and tests
  \item \texttt{README.md} explaining how to run the tests
  \item Screenshots of test execution results
  \item Brief reflection (200 words) on TDD experience
\end{enumerate}

\subsection{Extension Activities}
For advanced students who finish early:
\begin{itemize}
  \item Implement test coverage reporting with scoverage
  \item Add integration tests with embedded database
  \item Explore mutation testing with Stryker4s
  \item Implement custom ScalaCheck generators
\end{itemize}

\section{Troubleshooting}

\subsection{Common Issues}
\begin{itemize}
  \item \textbf{SBT dependency resolution}: Clear \texttt{~/.sbt} and \texttt{~/.ivy2} cache
  \item \textbf{Test not found}: Check package declarations and imports
  \item \textbf{Mockito issues}: Ensure proper trait/class mocking setup
  \item \textbf{ScalaCheck failures}: Use \texttt{whenever} to add preconditions
\end{itemize}

\subsection{Useful SBT Commands}
\begin{lstlisting}
sbt test                    # Run all tests
sbt testOnly *CalculatorSpec # Run specific test class
sbt "testOnly *Calculator* -- -z add" # Run specific test method
sbt test:console           # Start REPL with test classpath
sbt clean test             # Clean and run tests
\end{lstlisting}

\section{Resources and Further Reading}

\subsection{Documentation}
\begin{itemize}
  \item ScalaTest User Guide: \url{https://www.scalatest.org/user_guide}
  \item ScalaCheck Documentation: \url{https://scalacheck.org/documentation.html}
  \item Mockito Scala: \url{https://github.com/mockito/mockito-scala}
\end{itemize}

\subsection{Books and Articles}
\begin{itemize}
  \item "Test-Driven Development in Scala" by Kent Beck (adapted)
  \item "Property-Based Testing" by Fred Hebert
  \item "Effective Testing with ScalaTest" by Bill Venners
\end{itemize}

\subsection{Next Steps}
\begin{itemize}
  \item Integration testing with databases and HTTP
  \item Performance testing with ScalaMeter
  \item Contract testing with Pact
  \item Testing functional effects with ZIO Test
\end{itemize}

\end{document}
