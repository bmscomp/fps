\documentclass{beamer}
\usepackage[utf8]{inputenc}
\usepackage{booktabs} % For better table formatting
\usepackage{qrcode} % Main package for QR generation
\usepackage{graphicx} % For resizing
\usepackage{listings}
\usepackage[utf8]{inputenc}
\usepackage{xcolor}
\usepackage{multirow}
\usepackage{array}

% Define Scala syntax highlighting
\lstdefinestyle{scalaStyle}{
  language=Scala,
  basicstyle=\ttfamily\small,
  keywordstyle=\color{blue},
  commentstyle=\color{brown},
  stringstyle=\color{red},
  showstringspaces=false,
  breaklines=true,
  frame=single,
  numbers=left,
  numberstyle=\tiny\color{gray}
}

\definecolor{scalared}{RGB}{215,50,58}
\definecolor{darkblue}{RGB}{0,0,139}
\definecolor{pureblue}{RGB}{0,82,155}
\definecolor{impurered}{RGB}{186,0,13}
\definecolor{titleblue}{RGB}{0,82,155}
\definecolor{alertred}{RGB}{186,0,13}
\definecolor{darkgray}{RGB}{80,80,80}

\usetheme{Madrid}
\usecolortheme{default}

%------------------------------------------------------------
%This block of code defines the information to appear in the
%Title page
\title[Scala] %optional
{Functional programming in Scala}

\subtitle{Functions}

\author[Said BOUDJELDA] % (optional)
{Said BOUDJELDA}

\institute[efrei] % (optional)
{
  Senior Software Engineer @SCIAM\\
  Email : mohamed-said.boudjelda@intervenants.efrei.net \\ 
  Follow me on GitHub @bmscomp
}

\date[efrei 2025] % (optional)
{Course, May 2025}



% --- Add logo to left side of header ---
\logo{%
  \makebox[0.95\paperwidth]{%
    \includegraphics[height=0.8cm,keepaspectratio]{scala}%
    \hfill%  % Pushes logo to the left
  }%
}

\logo{\includegraphics[height=0.6cm]{logo}}


\newcommand{\chapterpage}[2]{
  \begin{frame}[plain]
    \centering
    \vfill
    {\usebeamerfont{title}\usebeamercolor[fg]{title}Chapter #1\\}
    \vspace{0.5cm}
    {\usebeamerfont{subtitle}#2}
    \vfill
  \end{frame}
}


%End of title page configuration block
%------------------------------------------------------------

\begin{document}

%The next statement creates the title page.
\frame{\titlepage}


%---------------------------------------------------------
%Changing visivility of the text
\begin{frame}
\frametitle{Definition}

In functional programming languages, functions are the fundamental building blocks of computation. They are treated differently than in imperative languages, more like mathematical functions and first-class citizens. 

We will learn here what that means in practice.

\end{frame}

%---------------------------------------------------------
\begin{frame}[fragile]
\frametitle{Impure functions}

\begin{lstlisting}[style=scalaStyle]
/**
 * Not pure because it have side affect 
 */
val greeting: Unit = println("Hello, Scala") 

\end{lstlisting}

\end{frame}



%---------------------------------------------------------
\begin{frame}[fragile]
\frametitle{Pure functions}

\begin{lstlisting}[style=scalaStyle]

/**
 * Deterministic: The same input always gives the same output.
 * Side effect-free: It does not modify any state or interact with the outside world (no I/O, * no global variable changes).
 */
val add = (x: Int, y: Int) => x + y

// Same Function with another declaration
val add: (Int, Int) => Int = (a, b) => a + b

\end{lstlisting}

\end{frame}

%---------------------------------------------------------
\begin{frame}[fragile]
\frametitle{Functions are not Methods}

Methods look and behave very similar to functions, but there are a few key differences between them.


\begin{lstlisting}[style=scalaStyle]

/**
 * Methods are defined with the def keyword. def is followed by a name, parameter list(s), a return type, and a body
 */
def add(x: Int, y: Int): Int = x + y
println(add(1, 2)) // 3

\end{lstlisting}

\end{frame}

%---------------------------------------------------------
\begin{frame}[fragile]
\frametitle{First-Class Citizens}

\begin{lstlisting}[style=scalaStyle]

/**
 * Be assigned to variables
 * Be passed as arguments
 * Be returned from other functions
 */

val sum = (x:Int , y: Int) => x + y
val double = (x: Int) => x * 2
def application(f: Int => Int, x: Int): Int = f(x)
def apply(f: (Int, Int) => Int, x: Int, y: Int): Int = f(x, y)

\end{lstlisting}

\end{frame}

%---------------------------------------------------------
\begin{frame}[fragile]
\frametitle{Function Composition}

\begin{lstlisting}[style=scalaStyle]

val f = (x: Int) => x + 1
val g = (x: Int) => x * 2
val h = f.compose(g) // h(x) = f(g(x)) = (x * 2) + 1

\end{lstlisting}

\end{frame}


%---------------------------------------------------------
\begin{frame}[fragile]
\frametitle{Lamdba  \( \lambda \) functions}

A lambda (aka anonymous function) is a function defined without a name


\begin{lstlisting}[style=scalaStyle]
(x: Int) => x * 2
\end{lstlisting}

\end{frame}

%---------------------------------------------------------
\begin{frame}[fragile]
\frametitle{Immutability and Referential Transparency}

\begin{lstlisting}[style=scalaStyle]
// Referentially transparent
def square(n: Int): Int = n * n
// Can be replaced with 16 + 16 or 32
val result = square(4) + square(4)  
// Not referentially transparent (depends on external state)

\end{lstlisting}

\end{frame}



%---------------------------------------------------------
\begin{frame}[fragile]
\frametitle{Recursion and tail call optimization}

Recursion is a function that it calls itself

\begin{lstlisting}[style=scalaStyle]
val factorial: Int => Int = (n: Int) => 
  if (n <= 1) 1          // Base case
  else n * factorial(n - 1) // Recursive case

// Tail call optimization
import scala.annotation.tailrec
val factorial: Int => Int = {
  @tailrec
  def loop(n: Int, acc: Int): Int = n match {
    case 0 | 1 => acc
    case _ => loop(n - 1, n * acc)
  }
  (n: Int) => loop(n, 1)
}

\end{lstlisting}
\end{frame}

%---------------------------------------------------------
\begin{frame}[fragile]
\frametitle{Higher-Order Functions (HOFs)}

Functions that take other functions as arguments or return them.

\end{frame}

%---------------------------------------------------------
\begin{frame}[fragile]
\frametitle{Higher-Order Functions (HOFs)}
\framesubtitle{Why Use HOFs?}

\begin{itemize}
    \item \textbf{Abstraction}: Reduce code duplication
    \item \textbf{Expressiveness}: Write more declarative code
    \item \textbf{Composability}: Combine small functions into complex operations
    \item \textbf{Flexibility} : Parameterize behavior
\end{itemize}

\end{frame}


%---------------------------------------------------------
\begin{frame}[fragile]
\frametitle{Higher-Order Functions (HOFs)}
\framesubtitle{map function}

The map function is a fundamental higher-order function in Scala that applies a given function to each element of a collection and returns a new collection with the transformed elements.

\begin{lstlisting}[style=scalaStyle]
/**
 * The map function functiona implementation
 */

val map: [A, B] => (List[A], A => B) => List[B] =
  [A, B] => (list, f) => list match
    case Nil => Nil
    case head :: tail => f(head) :: map(tail, f)

val numbers = List(1, 2, 3, 4)
val doubled = map(numbers, x => x * 2)  
//The result will be List(2, 4, 6, 8)

\end{lstlisting}

\end{frame}

%---------------------------------------------------------
\begin{frame}[fragile]
\frametitle{Higher-Order Functions (HOFs)}
\framesubtitle{flatMap function}

\textbf{flatMap} combines mapping and flattening operations. It's more powerful than map because it can handle nested structures and transform each element into a new collection, then flatten the results into a single collection

\begin{lstlisting}[style=scalaStyle]

val flatMap:[A, B] => (List[A], A => List[B]) => List[B] = 
  [A, B] => (list, f) => list.foldRight(List.empty[B])((a, acc) => f(a) ++ acc)

val numbers = List(1, 2, 3)
val result = flatMap(numbers)(n => List(n, n * 10))
// result: List[Int] = List(1, 10, 2, 20, 3, 30)

\end{lstlisting}

\end{frame}


%---------------------------------------------------------
\begin{frame}[fragile]
\frametitle{Higher-Order Functions (HOFs)}
\framesubtitle{map vs flatMap}
\centering
\begin{tabular}{lll}
\toprule
\textbf{Feature} & \texttt{map} & \texttt{flatMap} \\
\midrule
Return type & Single element & Collection/Monadic type \\
\addlinespace
Result structure & Wrapped in same context & Flattened structure \\
\addlinespace
Function signature & \texttt{A => B} & \texttt{A => F[B]} \\
\addlinespace
Use case & Simple transformations & Transform + flatten \\
\addlinespace
Nested collections & Preserves nesting & Removes one level of nesting \\
\addlinespace
For-comprehensions & Used with \texttt{yield} & Used with \texttt{<-} \\
\bottomrule
\end{tabular}

\vspace{1em}
\begin{itemize}
\item Both preserve the original collection
\item Both are higher-order functions
\item \texttt{flatMap} is monadic bind operation
\end{itemize}
\end{frame}



%---------------------------------------------------------
\begin{frame}[fragile]
\frametitle{Higher-Order Functions (HOFs)}
\framesubtitle{HOFs Filter function}

\begin{lstlisting}[style=scalaStyle]

/**
 * The filter function 
 */

val filter: List[Int] => (Int => Boolean) => List[Int] =
  list => predicate => list match
    case Nil => Nil
    case head :: tail =>
      if (predicate(head)) head :: filter(tail)(predicate)
      else filter(tail)(predicate)


\end{lstlisting}

\end{frame}

%---------------------------------------------------------
\begin{frame}[fragile]
\frametitle{Higher-Order Functions (HOFs)}
\framesubtitle{Fold function}

A \textbf{fold} function (also known as \textbf{reduce} or \textbf{aggregate}) is a powerful higher-order function in functional programming that reduces a collection (like a \textbf{List}) to a single value by applying a binary operation repeatedly, starting from an initial value.

\end{frame}


%---------------------------------------------------------
\begin{frame}[fragile]
\frametitle{HOFs/Fold  Function}
\framesubtitle{foldLeft implementation}

\textbf{foldLeft} processes elements left-to-right (from the first to the last).

\begin{lstlisting}[style=scalaStyle]

// foldLeft (Tail-recursive, stack-safe)
def left[A, B](list: List[A])(ini: B)(op: (B, A) => B): B =
  list match
    case Nil => initial
    case head :: tail => left(tail)(op(initial, head))(op)

val nums = List(1, 2, 3, 4)
// foldLeft: (((0 - 1) - 2) - 3) - 4 = -10
nums.foldLeft(0)(_ - _)  


\end{lstlisting}

\end{frame}

%---------------------------------------------------------
\begin{frame}[fragile]
\frametitle{HOFs Fold Function}
\framesubtitle{foldRight implementation}

\textbf{foldRight} processes elements right-to-left (from the last to the first).

\begin{lstlisting}[style=scalaStyle]

// foldRight (Not tail-recursive)
def foldRight[A, B](list: List[A])(initial: B)(op: (A, B) => B): B =
  list match
    case Nil => initial
    case head :: tail => op(head, foldRight(tail)(initial)(op))

val nums = List(1, 2, 3, 4)
// foldRight: 1 - (2 - (3 - (4 - 0))) = -2
nums.foldRight(0)(_ - _)  

\end{lstlisting}

\end{frame}

\begin{frame}{Folds  Performance \& Stack Safety Comparison}
Both foldLeft and foldRight are higher-order functions that combine elements of a collection into a single result, but they differ in key ways
\centering
\begin{tabular}{lcc}
\toprule
\textbf{Feature} & \textbf{\texttt{foldLeft}} & \textbf{\texttt{foldRight}} \\
\midrule
Tail-recursive & \checkmark & \texttimes \\
Stack-safe on large collections & \checkmark & \texttimes \\
Time Complexity & $O(n)$ (iterative) & $O(n)$ (but stack risk) \\
Space Complexity & $O(1)$ & $O(n)$ (call stack) \\
Works with infinite streams & \texttimes & \checkmark (if lazy) \\
\bottomrule
\end{tabular}

\vspace{1em}
\footnotesize
\begin{itemize}
\item \checkmark = Supported, \texttimes = Not supported
\item \texttt{foldLeft} uses constant space (tail-recursion optimized)
\item \texttt{foldRight} may overflow on large lists (non-tail-recursive)
\end{itemize}
\end{frame}

%---------------------------------------------------------
\begin{frame}[fragile]
\frametitle{Combining Immutability and Referential Transparency}

\begin{lstlisting}[style=scalaStyle]
def double(x: Int): Int = x * 2
def square(x: Int): Int = x * x
val numbers = List(1, 2, 3, 4, 5)
val processed = numbers.map(double).map(square)

// We can also compose
val composed = numbers.map(x => square(double(x))))

\end{lstlisting}
\end{frame}

%---------------------------------------------------------
\begin{frame}[fragile]
\frametitle{Partial functions}

A partial function is a function that is valid for only a subset of values of those types you might pass in to it


Do not confuse partial functions with partially applied functions
\begin{lstlisting}[style=scalaStyle]
val root: PartialFunction[Double,Double] = 
  case d if (d >= 0) => math.sqrt(d)

root(-1) // will result scala.MatchError
root.isDefinedAt(-1) // Will result false
root(3) // Double = 1.7320508075688772 

// List of only roots which are defined
List(0.5, -0.2, 4).collect(root) 

\end{lstlisting}
\end{frame}

%---------------------------------------------------------
\begin{frame}[fragile]
\frametitle{Partially Applied Functions}

A partially applied function is when you take a function with multiple parameters and fix some of them, creating a new function with fewer parameters.

\begin{lstlisting}[style=scalaStyle]
val add:(Int, Int, Int)=> Int = (a, b, c) => a + b + c 
// Partially applying it by fixing some parameters
val add2Numbers: (Int, Int) => Int = add(2, _, _)
val fivePlusSeven: Int => Int = add(5, 7, _)
// not partial application - all args applied
val fixedSum: Int = add(1, 2, 3) 
// Usage
addTwoToNumbers(3, 4)   // 2 + 3 + 4 = 9
fivePlusSeven(10)     // 5 + 7 + 10 = 22

\end{lstlisting}
\end{frame}

%---------------------------------------------------------
%Two columns
\begin{frame}
\frametitle{Partial Functions vs Partially applied Functions}

\begin{columns}

\column{0.5\textwidth}

\begin{itemize}
    \item Defined only for even inputs (partial domain)
    \item Uses pattern matching with guard condition
    \item Provides isDefinedAt to check applicability
    \item Can be combined with other partial functions
    \item Throws MatchError for undefined inputs
\end{itemize}
 
\column{0.5\textwidth}

\begin{itemize}
    \item Created from regular functions
    \item \textbf{Fixes} some parameters while leaving others open
    \item Results in a new function with fewer parameters
    \item Uses underscore \_ as placeholder for unspecified parameters
\end{itemize}

\end{columns}
\end{frame}

%---------------------------------------------------------
\begin{frame}[fragile]
\frametitle{Currying functions}

Currying is a functional programming technique that transforms a function taking multiple arguments into a sequence of functions that each take a single argument. 

\begin{lstlisting}[style=scalaStyle]
// Regular multi-argument function
val add: (Int, Int) => Int = (a, b) => a + b

// Curried version
val addCurried: Int => Int => Int = a => b => a + b

// Usage
val add5: Int => Int = addCurried(5)
println(add5(10))  // 15

\end{lstlisting}
\end{frame}

%---------------------------------------------------------
\begin{frame}[fragile]
\frametitle{Closures and Lexical Scope}

A closure is a function that "closes over" (captures) variables from its surrounding lexical scope, even if those variables are no longer in scope when the function is executed.

\begin{lstlisting}[style=scalaStyle]

val outer: Int => () => Int = x => {
  val captured = x
  () => captured + 10
}

val closure = outer(5)
println(closure())  // Output: 15

\end{lstlisting}
\end{frame}


%---------------------------------------------------------
\begin{frame}{Pure Functions: Key Papers}
\footnotesize

\begin{thebibliography}{9}
\beamertemplatetextbibitems

\subsection*{Theoretical Foundations}
\bibitem[Church, 1936]{church1936}
\textcolor{pureblue}{Church, A.} 
\textit{An Unsolvable Problem of Elementary Number Theory}. 1936.
\footnotesize{\\ \color{gray}Lambda calculus as basis for pure functions}

\bibitem[Backus, 1978]{backus1978}
\textcolor{pureblue}{Backus, J.} 
\textit{Can Programming Be Liberated from the von Neumann Style?} 1978.
\footnotesize{\\ \color{gray}FP manifesto emphasizing purity}

\subsection*{Language Design}
\bibitem[Hughes, 1989]{hughes1989}
\textcolor{pureblue}{Hughes, J.} 
\textit{Why Functional Programming Matters}. 1989.
\footnotesize{\\ \color{gray}Purity enables referential transparency}

\bibitem[Odersky, 2014]{odersky2014}
\textcolor{pureblue}{Odersky, M.} 
\textit{Scala: Unified OOP-FP}. 2014.
\footnotesize{\\ \color{gray}Implementing purity in impure environments}

\subsection*{Practical Applications}
\bibitem[Peyton Jones, 2003]{peytonjones2003}
\textcolor{pureblue}{Peyton Jones, S.} 
\textit{Haskell 98 Language Report}. 2003.
\footnotesize{\\ \color{gray}Pure-by-default design}

\bibitem[Cherny, 2020]{cherny2020}
\textcolor{pureblue}{Cherny, E.} 
\textit{Programming TypeScript}. 2020.
\footnotesize{\\ \color{gray}Practical purity in TypeScript/JavaScript}


\end{thebibliography}

\end{frame}

%---------------------------------------------------------

\begin{frame}[allowframebreaks]{Key Papers on Higher-Order Functions in Scala}
\footnotesize % Save space
\begin{thebibliography}{6} % Manual entries if no .bib file

\beamertemplatetextbibitems % Numeric labels

\subsection*{Foundations of Scala HOFs}
\bibitem{oderskym2004}
\color{darkblue}Odersky, M., et al.
\color{black}\textit{An Overview of the Scala Programming Language}, 2004.
\color{gray}\footnotesize (Introduces first-class HOFs in Scala's OOP/FP blend)

\bibitem{oliveira2010}
\color{darkblue}Oliveira, B. C. d. S., et al.
\color{black}\textit{Type Classes as Objects and Implicits}, OOPSLA 2010.
\color{gray}\footnotesize (HOFs + implicits for Haskell-style type classes)

\subsection*{Optimization \& Compilation}
\bibitem{kossakowski2012}
\color{darkblue}Kossakowski, G., et al.
\color{black}\textit{Miniboxing: Improving Specialization}, SCALA 2012.
\color{gray}\footnotesize (Optimizing generic HOFs via \texttt{@specialized})

\bibitem{burmako2013}
\color{darkblue}Burmako, E.
\color{black}\textit{Scala Macros: Let Our Powers Combine}, SCALA 2013.
\color{gray}\footnotesize (Macros for HOF inlining, e.g., \texttt{map}→\texttt{while})

\bibitem{brachthauser2020}
\color{darkblue}Brachthäuser, J. I.
\color{black}\textit{Effekt: Capability-Passing}, POPL 2020.
\color{gray}\footnotesize (HOFs for effect systems beyond monads)

\subsection*{Industry Case Studies}
\bibitem{twitter2011}
\color{darkblue}Twitter Engineering.
\color{black}\textit{Functional Programming at Twitter}, 2011.
\color{gray}\footnotesize (Scala's \texttt{map}/\texttt{flatMap} in distributed systems)

\end{thebibliography}
\end{frame}


\end{document}
%---------------------------------------------------------
