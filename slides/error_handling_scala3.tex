\documentclass{beamer}
\usepackage[utf8]{inputenc}
\usepackage{booktabs}
\usepackage{qrcode}
\usepackage{graphicx}
\usepackage{listings}
\usepackage{xcolor}
\usepackage{multirow}
\usepackage{array}

% Define Scala syntax highlighting
\lstdefinestyle{scalaStyle}{
  language=Scala,
  basicstyle=\ttfamily\small,
  keywordstyle=\color{blue},
  commentstyle=\color{brown},
  stringstyle=\color{red},
  showstringspaces=false,
  breaklines=true,
  frame=single,
  numbers=left,
  numberstyle=\tiny\color{gray}
}

\definecolor{scalared}{RGB}{215,50,58}
\definecolor{darkblue}{RGB}{0,0,139}
\definecolor{pureblue}{RGB}{0,82,155}
\definecolor{impurered}{RGB}{186,0,13}
\definecolor{titleblue}{RGB}{0,82,155}
\definecolor{alertred}{RGB}{186,0,13}
\definecolor{darkgray}{RGB}{80,80,80}

\usetheme{Madrid}
\usecolortheme{default}

\title[Scala 3 Error Handling]
{Error Handling in Scala 3}

\subtitle{Modern Functional Error Management}

\author[Said BOUDJELDA]
{Said BOUDJELDA}

\institute[efrei]
{
  Senior Software Engineer @SCIAM\\
  Email : mohamed-said.boudjelda@intervenants.efrei.net \\ 
  Follow me on GitHub @bmscomp
}

\date[efrei 2025]
{Course, May 2025}

\logo{\includegraphics[height=0.6cm]{logo}}

\begin{document}

\frame{\titlepage}

%---------------------------------------------------------
\begin{frame}
\frametitle{Overview}

Modern error handling in Scala 3 emphasizes:

\begin{itemize}
\item \textbf{Type Safety}: Errors visible in signatures
\item \textbf{Union Types}: Native error representation
\item \textbf{Enums}: Structured error hierarchies
\item \textbf{Composition}: Functional error chaining
\item \textbf{Performance}: Zero-cost abstractions
\end{itemize}

\textit{From exceptions to functional types}

\end{frame}

%---------------------------------------------------------
\begin{frame}[fragile]
\frametitle{Traditional vs Modern}

\begin{lstlisting}[style=scalaStyle]
// Old way: Exceptions (not type-safe)
def divide(a: Int, b: Int): Int = 
  if b == 0 then throw ArithmeticException("Division by zero")
  else a / b

// Scala 3 way: Union types (type-safe)
type Result[T] = T | String

def safeDivide(a: Int, b: Int): Result[Int] = 
  if b == 0 then "Division by zero" else a / b

// Usage
safeDivide(10, 2) match
  case result: Int => println(s"Success: $result")
  case error: String => println(s"Error: $error")
\end{lstlisting}

\end{frame}

%---------------------------------------------------------
\begin{frame}[fragile]
\frametitle{Union Types - The Modern Way}

\begin{lstlisting}[style=scalaStyle]
// Simple union types for errors
type ParseResult = Int | String

def parseNumber(s: String): ParseResult = 
  try s.toInt
  catch case _: NumberFormatException => s"Invalid: $s"

// Pattern matching is seamless
def processNumber(input: String) = parseNumber(input) match
  case num: Int => num * 2
  case error: String => 0

// Chaining operations
def calculate(x: String, y: String): ParseResult =
  (parseNumber(x), parseNumber(y)) match
    case (a: Int, b: Int) => a + b
    case (error: String, _) => error
    case (_, error: String) => error
\end{lstlisting}

\end{frame}

%---------------------------------------------------------
\begin{frame}[fragile]
\frametitle{Enums for Error Hierarchies}

\begin{lstlisting}[style=scalaStyle]
// Structured errors with enums
enum AppError(val message: String):
  case ValidationError(msg: String) extends AppError(msg)
  case NetworkError(msg: String) extends AppError(msg)
  case ParseError(msg: String) extends AppError(msg)
  
  def isRetryable: Boolean = this match
    case NetworkError(_) => true
    case _ => false

type ApiResult[T] = T | AppError

def validateAge(age: Int): ApiResult[Int] =
  if age >= 0 && age <= 150 then age
  else AppError.ValidationError(s"Invalid age: $age")

def parseAge(s: String): ApiResult[Int] =
  try validateAge(s.toInt)
  catch case _ => AppError.ParseError(s"Not a number: $s")
\end{lstlisting}

\end{frame}

%---------------------------------------------------------
\begin{frame}[fragile]
\frametitle{Extension Methods for Fluent APIs}

\begin{lstlisting}[style=scalaStyle]
// Extensions for composable error handling
extension [T](value: T)
  def validateThat(condition: T => Boolean, 
                   error: String): T | String =
    if condition(value) then value else error

extension [T](result: T | String)
  def andThen[U](f: T => U | String): U | String = result match
    case value: T => f(value)
    case error: String => error

// Fluent usage
def processInput(input: String): String | Int =
  input.validateThat(_.nonEmpty, "Input required")
    .andThen(_.validateThat(_.forall(_.isDigit), "Not a number"))
    .andThen(_.toInt)

println(processInput("123")) // 123
println(processInput("abc")) // Not a number
\end{lstlisting}

\end{frame}

%---------------------------------------------------------
\begin{frame}[fragile]
\frametitle{Option - Still Useful}

\begin{lstlisting}[style=scalaStyle]
// Option for null safety
def findUser(id: Int): Option[String] = 
  if id > 0 then Some(s"User$id") else None

// Functional operations
val result = findUser(1)
  .map(_.toUpperCase)
  .filter(_.startsWith("USER"))
  .getOrElse("Unknown")

// For-comprehension
def getUserData(id: Int, role: String) = for
  user <- findUser(id)
  validRole <- if role.nonEmpty then Some(role) else None
yield s"$user has role $validRole"

println(getUserData(1, "admin")) // Some(USER1 has role admin)
\end{lstlisting}

\end{frame}

%---------------------------------------------------------
\begin{frame}[fragile]
\frametitle{Either - When You Need Details}

\begin{lstlisting}[style=scalaStyle]
// Either for rich error information
def validateUser(name: String, age: Int): Either[String, String] = 
  for
    validName <- if name.nonEmpty then Right(name) 
                else Left("Name required")
    validAge <- if age > 0 then Right(age) 
               else Left("Age must be positive")
  yield s"$validName is $validAge years old"

// Error accumulation (custom approach)
def combineValidations[A, B](
  va: Either[String, A], 
  vb: Either[String, B]
): Either[List[String], (A, B)] = (va, vb) match
  case (Right(a), Right(b)) => Right((a, b))
  case (Left(e1), Left(e2)) => Left(List(e1, e2))
  case (Left(e), _) => Left(List(e))
  case (_, Left(e)) => Left(List(e))
\end{lstlisting}

\end{frame}

%---------------------------------------------------------
\begin{frame}[fragile]
\frametitle{Resource Management}

\begin{lstlisting}[style=scalaStyle]
import scala.util.Using

// Automatic resource cleanup
def readConfig(file: String): String | AppError = 
  Using(scala.io.Source.fromFile(file)) { source =>
    source.mkString
  }.toEither match
    case Right(content) => content
    case Left(ex) => AppError.NetworkError(ex.getMessage)

// Multiple resources
def copyFile(from: String, to: String): Unit | AppError = 
  Using.Manager { use =>
    val source = use(scala.io.Source.fromFile(from))
    val writer = use(java.io.PrintWriter(to))
    source.getLines().foreach(writer.println)
  }.toEither match
    case Right(_) => ()
    case Left(ex) => AppError.NetworkError(ex.getMessage)
\end{lstlisting}

\end{frame}

%---------------------------------------------------------
\begin{frame}[fragile]
\frametitle{Given/Using for Error Context}

\begin{lstlisting}[style=scalaStyle]
// Context for error handling
trait ErrorReporter:
  def report(error: String): Unit

given consoleReporter: ErrorReporter with
  def report(error: String): Unit = println(s"[ERROR] $error")

def validateWithContext[T](value: T, 
                          condition: T => Boolean,
                          errorMsg: String)
                          (using reporter: ErrorReporter): Option[T] =
  if condition(value) then Some(value)
  else 
    reporter.report(errorMsg)
    None

// Usage
validateWithContext(10, _ > 0, "Must be positive") // Some(10)
validateWithContext(-5, _ > 0, "Must be positive") // None + logs error
\end{lstlisting}

\end{frame}

%---------------------------------------------------------
\begin{frame}[fragile]
\frametitle{Async Error Handling}

\begin{lstlisting}[style=scalaStyle]
import scala.concurrent.Future

// Future with union types
def fetchUser(id: Int): Future[String | AppError] = 
  Future {
    if id > 0 then s"User$id"
    else AppError.ValidationError("Invalid ID")
  }

// Combining async operations
def getUserProfile(id: Int): Future[String | AppError] = 
  fetchUser(id).map {
    case user: String => s"Profile for $user"
    case error: AppError => error
  }

// Error recovery
def fetchUserWithRetry(id: Int): Future[String | AppError] = 
  fetchUser(id).recover {
    case ex: Exception => AppError.NetworkError(ex.getMessage)
  }
\end{lstlisting}

\end{frame}

%---------------------------------------------------------
\begin{frame}[fragile]
\frametitle{Pattern Matching Improvements}

\begin{lstlisting}[style=scalaStyle]
// Scala 3 pattern matching enhancements
def handleResult(result: String | Int | AppError): String = result match
  case s: String => s"Text: $s"
  case n: Int => s"Number: $n"
  case AppError.ValidationError(msg) => s"Validation: $msg"
  case AppError.NetworkError(msg) => s"Network: $msg"
  case AppError.ParseError(msg) => s"Parse: $msg"

// Guard patterns with union types
def categorizeValue(value: String | Int): String = value match
  case s: String if s.isEmpty => "Empty string"
  case s: String if s.length > 10 => "Long string"
  case s: String => "Short string"
  case n: Int if n < 0 => "Negative number"
  case n: Int => "Positive number"
\end{lstlisting}

\end{frame}

%---------------------------------------------------------
\begin{frame}[fragile]
\frametitle{Error Boundaries}

\begin{lstlisting}[style=scalaStyle]
// Error boundary pattern for isolation
def withErrorBoundary[T](operation: () => T): T | AppError =
  try operation()
  catch 
    case ex: IllegalArgumentException => 
      AppError.ValidationError(ex.getMessage)
    case ex: NumberFormatException => 
      AppError.ParseError(ex.getMessage)
    case ex: Exception => 
      AppError.NetworkError(ex.getMessage)

// Usage isolates errors
def riskyOperation(): String = throw RuntimeException("Boom!")

val result = withErrorBoundary(() => riskyOperation())
result match
  case value: String => println(s"Success: $value")
  case error: AppError => println(s"Contained error: ${error.message}")
\end{lstlisting}

\end{frame}

%---------------------------------------------------------
\begin{frame}[fragile]
\frametitle{Custom Validation DSL}

\begin{lstlisting}[style=scalaStyle]
// Build a validation DSL with Scala 3
case class ValidationResult[T](value: T, errors: List[String])

extension [T](value: T)
  def validate: ValidationResult[T] = ValidationResult(value, Nil)
  
extension [T](vr: ValidationResult[T])
  def check(condition: T => Boolean, error: String): ValidationResult[T] =
    if condition(vr.value) then vr
    else vr.copy(errors = vr.errors :+ error)
    
  def result: T | List[String] = 
    if vr.errors.isEmpty then vr.value else vr.errors

// Usage
def validateUser(name: String, age: Int) =
  (name, age).validate
    .check(_._1.nonEmpty, "Name required")
    .check(_._2 > 0, "Age must be positive")
    .check(_._2 < 150, "Age too high")
    .result
\end{lstlisting}

\end{frame}

%---------------------------------------------------------
\begin{frame}[fragile]
\frametitle{Testing Error Scenarios}

\begin{lstlisting}[style=scalaStyle]
// Simple testing approach
def testValidation(): Unit =
  // Test success
  val success = validateAge(25)
  assert(success == 25)
  
  // Test errors
  validateAge(-5) match
    case error: AppError.ValidationError => 
      assert(error.message.contains("Invalid"))
    case _ => sys.error("Expected validation error")
  
  // Test parsing
  parseAge("abc") match
    case AppError.ParseError(msg) => assert(msg.contains("number"))
    case _ => sys.error("Expected parse error")

// Property-based testing
def testProperty(): Unit =
  for i <- 1 to 100 do
    val age = scala.util.Random.nextInt(200) - 50
    validateAge(age) match
      case value: Int => assert(value >= 0 && value <= 150)
      case _: AppError => assert(age < 0 || age > 150)
\end{lstlisting}

\end{frame}

%---------------------------------------------------------
\begin{frame}[fragile]
\frametitle{Performance Considerations}

\begin{lstlisting}[style=scalaStyle]
// Union types are zero-cost
type FastResult = String | Int // No boxing!

// Avoid Option allocation in hot paths
def fastParse(s: String): FastResult =
  var i = 0
  var result = 0
  while i < s.length do
    val c = s.charAt(i)
    if c.isDigit then
      result = result * 10 + (c - '0')
      i += 1
    else
      return s"Invalid char at $i: $c"
  result

// Value classes for domain types
opaque type UserId = Int
object UserId:
  def apply(id: Int): UserId | String = 
    if id > 0 then id else "Invalid user ID"
  
extension (id: UserId) def value: Int = id
\end{lstlisting}

\end{frame}

%---------------------------------------------------------
\begin{frame}[fragile]
\frametitle{Real-World API Example}

\begin{lstlisting}[style=scalaStyle]
// Complete API with modern error handling
case class User(name: String, age: Int, email: String)

def createUser(data: Map[String, String]): User | List[String] =
  val validations = List(
    data.get("name").toRight("Missing name")
      .flatMap(n => if n.nonEmpty then Right(n) else Left("Empty name")),
    data.get("age").toRight("Missing age")
      .flatMap(a => try Right(a.toInt) catch case _ => Left("Invalid age")),
    data.get("email").toRight("Missing email")
      .flatMap(e => if e.contains("@") then Right(e) else Left("Invalid email"))
  )

  val errors = validations.collect { case Left(err) => err }
  if errors.nonEmpty then errors
  else 
    val Right(name) :: Right(age) :: Right(email) :: Nil = validations: @unchecked
    User(name, age, email)

// HTTP handler
def handleRequest(request: Map[String, String]): String =
  createUser(request) match
    case user: User => s"""{"status": "success", "user": "${user.name}"}"""
    case errors: List[String] => s"""{"status": "error", "errors": ${errors.mkString("[\"", "\", \"", "\"]")}}"""
\end{lstlisting}

\end{frame}

%---------------------------------------------------------
\begin{frame}[fragile]
\frametitle{Migration Strategy}

\begin{lstlisting}[style=scalaStyle]
// Step 1: Wrap existing exceptions
def legacyCode(): String = throw RuntimeException("Old code")

def wrapped(): String | String = 
  try legacyCode() catch case ex => ex.getMessage

// Step 2: Use union types
def modernVersion(): String | AppError =
  try legacyCode() 
  catch case ex => AppError.NetworkError(ex.getMessage)

// Step 3: Pure functional (no exceptions)
def pureVersion(input: String): String | AppError =
  input.validateThat(_.nonEmpty, "Input required")
    .andThen(processInput)

def processInput(s: String): String | String = 
  if s.startsWith("valid") then s"Processed: $s"
  else "Invalid input format"
\end{lstlisting}

\end{frame}

%---------------------------------------------------------
\begin{frame}[fragile]
\frametitle{Monitoring and Logging}

\begin{lstlisting}[style=scalaStyle]
// Structured error tracking
def logError(error: AppError, context: Map[String, String] = Map.empty): Unit =
  val logData = Map(
    "error_type" -> error.getClass.getSimpleName,
    "message" -> error.message,
    "timestamp" -> java.time.Instant.now().toString
  ) ++ context
  
  println(s"ERROR: ${logData.map((k, v) => s"$k=$v").mkString(", ")}")

// Usage with error handling
def processWithLogging[T](operation: String)(thunk: => T | AppError): T | AppError =
  val startTime = System.currentTimeMillis()
  thunk match
    case result: T => 
      println(s"$operation completed in ${System.currentTimeMillis() - startTime}ms")
      result
    case error: AppError => 
      logError(error, Map("operation" -> operation, "duration" -> s"${System.currentTimeMillis() - startTime}ms"))
      error

val result = processWithLogging("user-creation") {
  createUser(Map("name" -> "", "age" -> "invalid"))
}
\end{lstlisting}

\end{frame}

%---------------------------------------------------------
\begin{frame}[fragile]
\frametitle{Error Handling Patterns Summary}

\begin{table}[ht]
\centering
\small
\begin{tabular}{|l|l|l|}
\hline
\textbf{Pattern} & \textbf{Use Case} & \textbf{Scala 3 Feature} \\
\hline
Union Types & Simple errors & Native type system \\
\hline
Enums & Error hierarchies & Structured ADTs \\
\hline
Option & Null safety & Built-in monad \\
\hline
Either & Rich errors & Right-biased \\
\hline
Extensions & Fluent APIs & Extension methods \\
\hline
Given/Using & Error context & Context functions \\
\hline
\end{tabular}
\end{table}

\textbf{Recommendation:} Start with union types, add structure with enums

\end{frame}

%---------------------------------------------------------
\begin{frame}[fragile]
\frametitle{Best Practices}

\begin{enumerate}
\item \textbf{Union types first} for simple error cases
\item \textbf{Enums} for structured error hierarchies  
\item \textbf{Extension methods} for composable APIs
\item \textbf{Pattern matching} over exception handling
\item \textbf{Value types} for performance-critical code
\item \textbf{Test error paths} as thoroughly as success paths
\end{enumerate}

\begin{lstlisting}[style=scalaStyle]
// ✅ Good: Type-safe and composable
def parseConfig(file: String): Config | AppError

// ❌ Avoid: Hidden exceptions
def parseConfig(file: String): Config // can throw!

// ✅ Good: Composable validation
input.validateThat(_.nonEmpty, "Required")
  .andThen(parseNumber)
  .andThen(validateRange)
\end{lstlisting}

\end{frame}

%---------------------------------------------------------
\begin{frame}[fragile]
\frametitle{Key Takeaways}

\begin{itemize}
\item \textbf{Union types} are the modern Scala 3 way
\item \textbf{Zero runtime cost} compared to exceptions
\item \textbf{Type safety} prevents runtime surprises
\item \textbf{Composition} enables building complex validations
\item \textbf{Pattern matching} provides elegant error handling
\item \textbf{Extensions} create fluent, readable APIs
\end{itemize}

\vspace{1em}
\textit{Embrace Scala 3's type system for robust, efficient error handling}

\end{frame}

%---------------------------------------------------------
\begin{frame}{References}
\footnotesize

\begin{thebibliography}{9}
\beamertemplatetextbibitems

\subsection*{Scala 3 Documentation}
\bibitem[Odersky, 2023]{odersky2023}
\textcolor{pureblue}{Odersky, M.} 
\textit{Scala 3 Reference: Union Types}. 2023.

\bibitem[EPFL, 2023]{epfl2023}
\textcolor{pureblue}{EPFL Team} 
\textit{Scala 3 Book: Error Handling}. 2023.

\subsection*{Best Practices}
\bibitem[Wampler, 2021]{wampler2021}
\textcolor{pureblue}{Wampler, D.} 
\textit{Programming Scala 3: Functional Error Handling}. 2021.

\bibitem[Spiewak, 2020]{spiewak2020}
\textcolor{pureblue}{Spiewak, D.} 
\textit{Error Handling in Modern Scala}. 2020.

\end{thebibliography}

\end{frame}

\end{document}
