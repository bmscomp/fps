\documentclass{beamer}
\usepackage[utf8]{inputenc}
\usepackage{booktabs}
\usepackage{qrcode}
\usepackage{graphicx}
\usepackage{listings}
\usepackage{xcolor}
\usepackage{multirow}
\usepackage{array}

% Define Scala syntax highlighting
\lstdefinestyle{scalaStyle}{
  language=Scala,
  basicstyle=\ttfamily\small,
  keywordstyle=\color{blue},
  commentstyle=\color{brown},
  stringstyle=\color{red},
  showstringspaces=false,
  breaklines=true,
  frame=single,
  numbers=left,
  numberstyle=\tiny\color{gray}
}

\definecolor{scalared}{RGB}{215,50,58}
\definecolor{darkblue}{RGB}{0,0,139}
\definecolor{pureblue}{RGB}{0,82,155}
\definecolor{impurered}{RGB}{186,0,13}
\definecolor{titleblue}{RGB}{0,82,155}
\definecolor{alertred}{RGB}{186,0,13}
\definecolor{darkgray}{RGB}{80,80,80}

\usetheme{Madrid}
\usecolortheme{default}

\title[Testing in Scala]
{Functional programming with Scala}

\subtitle{Unit and Integration Testing}

\author[Said BOUDJELDA]
{Said BOUDJELDA}

\institute[efrei]
{
  Senior Software Engineer @SCIAM\\
  GitHub @bmscomp
}

\date[efrei 2025]
{efrei, 2025}

\logo{\includegraphics[height=0.6cm]{logo}}

\begin{document}

%---------------------------------------------------------
\frame{\titlepage}

%---------------------------------------------------------
\begin{frame}
\frametitle{Table of Contents}
\tableofcontents
\end{frame}

%---------------------------------------------------------
\section{Introduction to Testing}

\begin{frame}
\frametitle{Why Testing Matters}

\begin{itemize}
  \item \textbf{Confidence}: Code works as expected
  \item \textbf{Regression Prevention}: Catch breaking changes
  \item \textbf{Documentation}: Tests describe behavior
  \item \textbf{Refactoring Safety}: Change code without fear
  \item \textbf{Design Feedback}: Tests reveal design issues
\end{itemize}

\vspace{0.3cm}
\textit{Good tests are your safety net}

\end{frame}

%---------------------------------------------------------
\begin{frame}
\frametitle{Testing Pyramid}

\begin{center}
\begin{tabular}{|c|c|c|}
\hline
\textbf{Level} & \textbf{Speed} & \textbf{Scope} \\
\hline
Unit Tests & Fast & Single function/class \\
\hline
Integration Tests & Medium & Multiple components \\
\hline
E2E Tests & Slow & Full system \\
\hline
\end{tabular}
\end{center}

\vspace{0.5cm}
\textbf{Rule}: More unit tests, fewer integration tests, minimal E2E

\end{frame}

%---------------------------------------------------------
\section{Scala 3 Testing Frameworks}

\begin{frame}
\frametitle{Popular Testing Frameworks}

\begin{itemize}
  \item \textbf{ScalaTest}: Most popular, many styles
  \item \textbf{MUnit}: Lightweight, fast compilation
  \item \textbf{uTest}: Minimal boilerplate
  \item \textbf{Specs2}: BDD-style specifications
  \item \textbf{ZIO Test}: For ZIO applications
\end{itemize}

\vspace{0.3cm}
\textit{We'll focus on ScalaTest and MUnit}

\end{frame}

%---------------------------------------------------------
\begin{frame}[fragile]
\frametitle{ScalaTest Setup}

\begin{lstlisting}[style=scalaStyle]
// build.sbt
libraryDependencies += "org.scalatest" %% "scalatest" % "3.2.15" % Test

class CalculatorSpec extends AnyFlatSpec with Matchers:
  "A Calculator" should "add two numbers" in {
    val result = Calculator.add(2, 3)
    result should be(5)
  }
\end{lstlisting}

Basic ScalaTest structure

\end{frame}

%---------------------------------------------------------
\section{Unit Testing}

\begin{frame}[fragile]
\frametitle{Testing Pure Functions}

\begin{lstlisting}[style=scalaStyle]
object MathUtils:
  def factorial(n: Int): Int = 
    if n <= 1 then 1 else n * factorial(n - 1)

class MathUtilsSpec extends AnyFlatSpec with Matchers:
  "factorial" should "calculate correctly" in {
    MathUtils.factorial(5) should be(120)
  }
\end{lstlisting}

Testing pure functions is straightforward

\end{frame}

%---------------------------------------------------------
\begin{frame}[fragile]
\frametitle{Property-Based Testing}

\begin{lstlisting}[style=scalaStyle]
import org.scalatestplus.scalacheck.ScalaCheckPropertyChecks

class MathPropertiesSpec extends AnyPropSpec with ScalaCheckPropertyChecks:
  property("addition is commutative") {
    forAll { (a: Int, b: Int) =>
      MathUtils.add(a, b) should equal(MathUtils.add(b, a))
    }
  }
\end{lstlisting}

Generate test cases automatically

\end{frame}

%---------------------------------------------------------
\begin{frame}[fragile]
\frametitle{Testing with Mocks}

\begin{lstlisting}[style=scalaStyle]
trait UserRepository:
  def findById(id: Long): Option[User]

class UserServiceSpec extends AnyFlatSpec with MockitoSugar:
  "UserService" should "find user" in {
    val mockRepo = mock[UserRepository]
    when(mockRepo.findById(1L)).thenReturn(Some(user))
  }
\end{lstlisting}

\end{frame}

%---------------------------------------------------------
\section{Integration Testing}

\begin{frame}
\frametitle{Integration Testing Concepts}

\begin{itemize}
  \item \textbf{Database Integration}: Test with real/embedded DB
  \item \textbf{HTTP Integration}: Test REST API endpoints
  \item \textbf{Message Queues}: Test async communication
  \item \textbf{External Services}: Test third-party integrations
  \item \textbf{Container Testing}: Use Testcontainers
\end{itemize}

\vspace{0.3cm}
\textit{Integration tests verify component interaction}

\end{frame}

%---------------------------------------------------------
\begin{frame}[fragile]
\frametitle{Database Integration Testing}

\begin{lstlisting}[style=scalaStyle]
class UserRepositoryIntegrationSpec extends AnyFlatSpec:
  val db = Database.forConfig("h2mem")
  val userRepo = new UserRepository(db)
  
  "UserRepository" should "save users" in {
    val user = User(0L, "Alice", "alice@test.com")
    val saved = Await.result(userRepo.save(user), 2.seconds)
    saved.name should be("Alice")
  }
\end{lstlisting}

\end{frame}

%---------------------------------------------------------
\begin{frame}[fragile]
\frametitle{HTTP API Testing}

\begin{lstlisting}[style=scalaStyle]
class UserApiSpec extends AnyFlatSpec with ScalatestRouteTest:
  val userService = mock[UserService]
  val routes = new UserRoutes(userService).routes
  
  "UserAPI" should "create user" in {
    Post("/users", userData.parseJson) ~> routes ~> check {
      status should be(StatusCodes.Created)
    }
  }
\end{lstlisting}

Test HTTP endpoints in isolation

\end{frame}

%---------------------------------------------------------
\begin{frame}[fragile]
\frametitle{Testcontainers Integration}

\begin{lstlisting}[style=scalaStyle]
import com.dimafeng.testcontainers.PostgreSQLContainer
import com.dimafeng.testcontainers.scalatest.TestContainerForEach

class PostgresIntegrationSpec extends AnyFlatSpec 
  with TestContainerForEach:
  
  override val containerDef = PostgreSQLContainer.Def()
  
  "PostgreSQL" should "work with real database" in {
    withContainers { postgres =>
      val userRepo = new UserRepository(postgres.jdbcUrl)
      val result = Await.result(userRepo.save(user), 3.seconds)
    }
  }
\end{lstlisting}

\end{frame}

%---------------------------------------------------------
\section{Testing Best Practices}

\begin{frame}
\frametitle{Testing Best Practices}

\begin{itemize}
  \item \textbf{AAA Pattern}: Arrange, Act, Assert
  \item \textbf{One Assertion}: Test one thing at a time
  \item \textbf{Descriptive Names}: Clear test method names
  \item \textbf{Independent Tests}: No test interdependence
  \item \textbf{Fast Execution}: Keep unit tests fast
  \item \textbf{Test Data Builders}: Use factories for setup
  \item \textbf{Clean Resources}: Proper setup/teardown
\end{itemize}

\vspace{0.3cm}
\textit{Good tests are readable and maintainable}

\end{frame}

%---------------------------------------------------------
\begin{frame}[fragile]
\frametitle{Test Data Builders}

\begin{lstlisting}[style=scalaStyle]
class UserBuilder:
  private var name: String = "John Doe"
  def withName(name: String): UserBuilder = { this.name = name; this }
  def build(): User = User(1L, name, "john@example.com")

// Usage: UserBuilder().withName("Alice").build()
\end{lstlisting}

\end{frame}

%---------------------------------------------------------
\section{Advanced Testing Topics}

\begin{frame}[fragile]
\frametitle{Testing Async Code}

\begin{lstlisting}[style=scalaStyle]
class AsyncServiceSpec extends AnyFlatSpec with ScalaFutures:
  "AsyncService" should "handle futures" in {
    val service = new AsyncUserService()
    val futureResult = service.findUserAsync(1L)
    
    whenReady(futureResult) { user =>
      user.name should be("Alice")
    }
  }
\end{lstlisting}

Handle asynchronous operations properly

\end{frame}

%---------------------------------------------------------
\begin{frame}[fragile]
\frametitle{Testing with ZIO}

\begin{lstlisting}[style=scalaStyle]
object UserServiceSpec extends ZIOSpecDefault:
  def spec = suite("UserService")(
    test("should create user") {
      for {
        service <- ZIO.service[UserService]
        user    <- service.createUser("Alice", "alice@test.com")
      } yield assertTrue(user.name == "Alice")
    }
  ).provide(UserServiceLive.layer)
\end{lstlisting}

ZIO Test provides functional testing

\end{frame}

%---------------------------------------------------------
\begin{frame}
\frametitle{Test-Driven Development (TDD)}

\begin{itemize}
  \item \textbf{Red}: Write failing test first
  \item \textbf{Green}: Write minimal code to pass
  \item \textbf{Refactor}: Improve code while keeping tests green
  \item \textbf{Benefits}: Better design, 100\% test coverage
  \item \textbf{Challenges}: Requires discipline and practice
\end{itemize}

\vspace{0.3cm}
\textit{TDD drives better software design}

\end{frame}

%---------------------------------------------------------
\begin{frame}[fragile]
\frametitle{TDD Example}

\begin{lstlisting}[style=scalaStyle]
// 1. Red - Write failing test
class StringUtilsSpec extends AnyFlatSpec:
  "reverse" should "reverse a string" in {
    StringUtils.reverse("hello") should be("olleh")
  }

// 2. Green - Make it pass
object StringUtils:
  def reverse(s: String): String = s.reverse
\end{lstlisting}

Write test first, then implementation

\end{frame}

%---------------------------------------------------------
\begin{frame}
\frametitle{Behavior-Driven Development (BDD)}

\begin{itemize}
  \item \textbf{Given-When-Then}: Structure test scenarios
  \item \textbf{Readable Tests}: Business-friendly language
  \item \textbf{Collaboration}: Bridge between technical and business
  \item \textbf{Living Documentation}: Tests as specifications
  \item \textbf{ScalaTest Support}: Feature and scenario DSL
\end{itemize}

\vspace{0.3cm}
\textit{BDD focuses on behavior and outcomes}

\end{frame}

%---------------------------------------------------------
\begin{frame}[fragile]
\frametitle{BDD with ScalaTest}

\begin{lstlisting}[style=scalaStyle]
class UserRegistrationSpec extends FeatureSpec with GivenWhenThen:
  feature("User Registration") {
    scenario("Valid user signs up") {
      given("a valid email address")
      val email = "user@example.com"
      
      when("user registers")
      val result = UserService.register(email)
      
      then("user should be created")
      result should be(defined)
    }
  }
\end{lstlisting}

\end{frame}

%---------------------------------------------------------
\begin{frame}
\frametitle{Test Doubles}

\begin{itemize}
  \item \textbf{Mocks}: Verify interactions and behavior
  \item \textbf{Stubs}: Return predefined responses
  \item \textbf{Fakes}: Working implementations for testing
  \item \textbf{Spies}: Record calls to real objects
  \item \textbf{Dummies}: Objects passed but never used
\end{itemize}

\vspace{0.3cm}
\textbf{Choose the right test double for your needs}

\end{frame}

%---------------------------------------------------------
\begin{frame}[fragile]
\frametitle{Stubs vs Mocks}

\begin{lstlisting}[style=scalaStyle]
// Stub - returns data
val stubRepo = stub[UserRepository]
when(stubRepo.findById(1L)).thenReturn(Some(user))

// Mock - verifies behavior
val mockRepo = mock[UserRepository]
service.deleteUser(1L)
verify(mockRepo).delete(1L)
\end{lstlisting}

Stubs provide data, mocks verify behavior

\end{frame}

%---------------------------------------------------------
\begin{frame}[fragile]
\frametitle{Parameterized Tests}

\begin{lstlisting}[style=scalaStyle]
class MathSpec extends AnyFlatSpec with TableDrivenPropertyChecks:
  val examples = Table(
    ("input", "expected"),
    (0, 1), (1, 1), (5, 120)
  )
  
  forAll(examples) { (input, expected) =>
    MathUtils.factorial(input) should be(expected)
  }
\end{lstlisting}

Test multiple inputs efficiently

\end{frame}

%---------------------------------------------------------
\begin{frame}[fragile]
\frametitle{Test Fixtures}

\begin{lstlisting}[style=scalaStyle]
trait UserFixture:
  def withUsers[T](testCode: List[User] => T): T = {
    val users = List(
      User(1L, "Alice", "alice@test.com"),
      User(2L, "Bob", "bob@test.com")
    )
    testCode(users)
  }

class UserServiceSpec extends AnyFlatSpec with UserFixture:
  "UserService" should "process users" in withUsers { users =>
    // Test with fixture data
  }
\end{lstlisting}

\end{frame}

%---------------------------------------------------------
\begin{frame}
\frametitle{Performance Testing}

\begin{itemize}
  \item \textbf{Load Testing}: Normal expected load
  \item \textbf{Stress Testing}: Beyond normal capacity
  \item \textbf{Spike Testing}: Sudden load increases
  \item \textbf{Volume Testing}: Large amounts of data
  \item \textbf{Tools}: JMeter, Gatling, ScalaMeter
\end{itemize}

\vspace{0.3cm}
\textit{Ensure your application scales}

\end{frame}

%---------------------------------------------------------
\begin{frame}[fragile]
\frametitle{ScalaMeter Example}

\begin{lstlisting}[style=scalaStyle]
import org.scalameter._

class PerformanceSpec extends AnyFlatSpec:
  "List operations" should "be performant" in {
    val sizes = Gen.range("size")(1000, 5000, 1000)
    
    val time = measure {
      val list = (1 to 1000).toList
      list.map(_ * 2).filter(_ > 100)
    }
    
    time should be < 10.millis
  }
\end{lstlisting}

\end{frame}

%---------------------------------------------------------
\begin{frame}
\frametitle{Mutation Testing}

\begin{itemize}
  \item \textbf{Concept}: Introduce bugs to test your tests
  \item \textbf{Mutants}: Modified versions of your code
  \item \textbf{Kill Rate}: Percentage of mutants caught
  \item \textbf{Benefits}: Reveals weak tests
  \item \textbf{Tools}: Stryker4s, PIT for Scala
\end{itemize}

\vspace{0.3cm}
\textbf{Test quality matters more than coverage}

\end{frame}

%---------------------------------------------------------
\begin{frame}
\frametitle{Contract Testing}

\begin{itemize}
  \item \textbf{Consumer-Driven}: Consumers define expectations
  \item \textbf{Provider Testing}: Verify contract compliance
  \item \textbf{Independent Deployment}: Test service boundaries
  \item \textbf{Microservices}: Essential for distributed systems
  \item \textbf{Tools}: Pact, Spring Cloud Contract
\end{itemize}

\vspace{0.3cm}
\textit{Test service interactions safely}

\end{frame}

%---------------------------------------------------------
\begin{frame}[fragile]
\frametitle{Snapshot Testing}

\begin{lstlisting}[style=scalaStyle]
class ApiResponseSpec extends AnyFlatSpec:
  "API" should "return expected JSON" in {
    val response = api.getUser(1L)
    val json = response.asJson.spaces2
    
    // Compare with saved snapshot
    json should matchSnapshot("user-response.json")
  }
\end{lstlisting}

Capture and compare complex outputs

\end{frame}

%---------------------------------------------------------
\begin{frame}
\frametitle{Testing Microservices}

\begin{itemize}
  \item \textbf{Service Tests}: Individual service testing
  \item \textbf{Contract Tests}: API compatibility
  \item \textbf{Component Tests}: Service with dependencies
  \item \textbf{End-to-End Tests}: Full system scenarios
  \item \textbf{Chaos Testing}: Failure resilience
\end{itemize}

\vspace{0.3cm}
\textbf{Layer your testing strategy}

\end{frame}

%---------------------------------------------------------
\begin{frame}[fragile]
\frametitle{Testing Configuration}

\begin{lstlisting}[style=scalaStyle]
// application-test.conf
database {
  url = "jdbc:h2:mem:test"
  driver = "org.h2.Driver"
}

class ConfigSpec extends AnyFlatSpec:
  "Config" should "load test settings" in {
    val config = ConfigFactory.load("application-test")
    config.getString("database.url") should include("h2:mem")
  }
\end{lstlisting}

Separate test configuration

\end{frame}

%---------------------------------------------------------
\begin{frame}
\frametitle{Continuous Testing}

\begin{itemize}
  \item \textbf{Fast Feedback}: Run tests on every change
  \item \textbf{Test Categories}: Unit, integration, acceptance
  \item \textbf{Parallel Execution}: Speed up test suites
  \item \textbf{Flaky Test Management}: Identify and fix unstable tests
  \item \textbf{Test Reporting}: Visibility and metrics
\end{itemize}

\vspace{0.3cm}
\textit{Automate testing in your pipeline}

\end{frame}

%---------------------------------------------------------
\begin{frame}
\frametitle{Test Organization}

\begin{itemize}
  \item \textbf{Directory Structure}: Mirror production code
  \item \textbf{Naming Conventions}: Consistent test naming
  \item \textbf{Test Categories}: Use tags and suites
  \item \textbf{Shared Utilities}: Common test helpers
  \item \textbf{Test Resources}: Manage test data files
\end{itemize}

\vspace{0.3cm}
\textbf{Organize tests for maintainability}

\end{frame}

%---------------------------------------------------------
\begin{frame}[fragile]
\frametitle{Error Handling in Tests}

\begin{lstlisting}[style=scalaStyle]
class ErrorHandlingSpec extends AnyFlatSpec:
  "Service" should "handle errors gracefully" in {
    assertThrows[ValidationException] {
      UserService.createUser("", "invalid-email")
    }
  }
  
  it should "return proper error messages" in {
    val result = UserService.validateEmail("invalid")
    result.left.value should include("Invalid email")
  }
\end{lstlisting}

\end{frame}

%---------------------------------------------------------
\begin{frame}
\frametitle{Testing Anti-Patterns}

\begin{itemize}
  \item \textbf{Ice Cream Cone}: Too many E2E tests
  \item \textbf{Testing Implementation}: Test behavior, not internals
  \item \textbf{Fragile Tests}: Brittle tests that break easily
  \item \textbf{Slow Tests}: Long-running test suites
  \item \textbf{Mystery Guest}: Hidden test dependencies
\end{itemize}

\vspace{0.3cm}
\textit{Avoid these common testing mistakes}

\end{frame}

%---------------------------------------------------------
\section{Testing Strategy}

\begin{frame}
\frametitle{Test Coverage and Metrics}

\begin{itemize}
  \item \textbf{Line Coverage}: Percentage of code executed
  \item \textbf{Branch Coverage}: All decision paths tested
  \item \textbf{Mutation Testing}: Test quality assessment
  \item \textbf{Performance Testing}: Load and stress testing
  \item \textbf{Security Testing}: Vulnerability assessment
\end{itemize}

\vspace{0.3cm}
\textbf{Tools}: scoverage, sbt-scoverage, ScalaMeter

\end{frame}

%---------------------------------------------------------
\begin{frame}
\frametitle{Continuous Integration}

\begin{itemize}
  \item \textbf{Automated Testing}: Run tests on every commit
  \item \textbf{Parallel Execution}: Speed up test suites
  \item \textbf{Test Categories}: Unit, integration, acceptance
  \item \textbf{Quality Gates}: Coverage thresholds
  \item \textbf{Test Reports}: Visibility into test results
\end{itemize}

\vspace{0.3cm}
\textit{CI/CD pipeline ensures code quality}

\end{frame}

%---------------------------------------------------------
\begin{frame}
\frametitle{Conclusion}

\begin{itemize}
  \item \textbf{Test Early}: Write tests as you code
  \item \textbf{Test Often}: Automate everything
  \item \textbf{Test Smart}: Focus on critical paths
  \item \textbf{Test Fast}: Keep feedback cycles short
  \item \textbf{Test Real}: Use integration tests strategically
\end{itemize}

\vspace{0.5cm}
\textbf{Remember}: Good tests enable confident refactoring and feature development

\end{frame}

%---------------------------------------------------------
\begin{frame}
\frametitle{Thank You!}

\begin{center}
\Large Questions?

\vspace{1cm}

\textbf{Said BOUDJELDA}\\
Senior Software Engineer @SCIAM\\
GitHub: @bmscomp

\vspace{0.5cm}
\qrcode[height=2cm]{https://github.com/bmscomp}
\end{center}

\end{frame}

\end{document}
